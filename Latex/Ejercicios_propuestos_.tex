%%%%%%%%%%%%  Generated using docx2latex.com  %%%%%%%%%%%%%%

%%%%%%%%%%%%  v2.0.0-beta  %%%%%%%%%%%%%%

\documentclass[12pt]{article}
\usepackage{amsmath}
\usepackage{latexsym}
\usepackage{amsfonts}
\usepackage[normalem]{ulem}
\usepackage{soul}
\usepackage{array}
\usepackage{amssymb}
\usepackage{extarrows}
\usepackage{graphicx}
\usepackage[backend=biber,
style=numeric,
sorting=none,
isbn=false,
doi=false,
url=false,
]{biblatex}\addbibresource{bibliography.bib}

\usepackage{subfig}
\usepackage{wrapfig}
\usepackage{wasysym}
\usepackage{enumitem}
\usepackage{adjustbox}
\usepackage{ragged2e}
\usepackage[svgnames,table]{xcolor}
\usepackage{tikz}
\usepackage{longtable}
\usepackage{changepage}
\usepackage{setspace}
\usepackage{hhline}
\usepackage{multicol}
\usepackage{tabto}
\usepackage{float}
\usepackage{multirow}
\usepackage{makecell}
\usepackage{fancyhdr}
\usepackage[toc,page]{appendix}
\usepackage[hidelinks]{hyperref}
\usetikzlibrary{shapes.symbols,shapes.geometric,shadows,arrows.meta}
\tikzset{>={Latex[width=1.5mm,length=2mm]}}
\usepackage{flowchart}\usepackage[paperheight=11.69in,paperwidth=8.27in,left=0.87in,right=0.87in,top=0.98in,bottom=0.79in,headheight=1in]{geometry}
\usepackage[utf8]{inputenc}
\usepackage[T1]{fontenc}
\TabPositions{0.5in,1.0in,1.5in,2.0in,2.5in,3.0in,3.5in,4.0in,4.5in,5.0in,5.5in,6.0in,6.5in,}

\urlstyle{same}

\renewcommand{\_}{\kern-1.5pt\textunderscore\kern-1.5pt}

 %%%%%%%%%%%%  Set Depths for Sections  %%%%%%%%%%%%%%

% 1) Section
% 1.1) SubSection
% 1.1.1) SubSubSection
% 1.1.1.1) Paragraph
% 1.1.1.1.1) Subparagraph


\setcounter{tocdepth}{5}
\setcounter{secnumdepth}{5}


 %%%%%%%%%%%%  Set Depths for Nested Lists created by \begin{enumerate}  %%%%%%%%%%%%%%


\setlistdepth{9}
\renewlist{enumerate}{enumerate}{9}
		\setlist[enumerate,1]{label=\arabic*)}
		\setlist[enumerate,2]{label=\alph*)}
		\setlist[enumerate,3]{label=(\roman*)}
		\setlist[enumerate,4]{label=(\arabic*)}
		\setlist[enumerate,5]{label=(\Alph*)}
		\setlist[enumerate,6]{label=(\Roman*)}
		\setlist[enumerate,7]{label=\arabic*}
		\setlist[enumerate,8]{label=\alph*}
		\setlist[enumerate,9]{label=\roman*}

\renewlist{itemize}{itemize}{9}
		\setlist[itemize]{label=$\cdot$}
		\setlist[itemize,1]{label=\textbullet}
		\setlist[itemize,2]{label=$\circ$}
		\setlist[itemize,3]{label=$\ast$}
		\setlist[itemize,4]{label=$\dagger$}
		\setlist[itemize,5]{label=$\triangleright$}
		\setlist[itemize,6]{label=$\bigstar$}
		\setlist[itemize,7]{label=$\blacklozenge$}
		\setlist[itemize,8]{label=$\prime$}

\setlength{\topsep}{0pt}\setlength{\parindent}{0pt}

 %%%%%%%%%%%%  This sets linespacing (verticle gap between Lines) Default=1 %%%%%%%%%%%%%%


\renewcommand{\arraystretch}{1.3}


%%%%%%%%%%%%%%%%%%%% Document code starts here %%%%%%%%%%%%%%%%%%%%



\begin{document}


%%%%%%%%%%%%%%%%%%%% Figure/Image No: 1 starts here %%%%%%%%%%%%%%%%%%%%

\begin{figure}[H]
	\begin{Center}
		\includegraphics[width=5.49in,height=1.58in]{./media/image1.jpeg}
	\end{Center}
\end{figure}


%%%%%%%%%%%%%%%%%%%% Figure/Image No: 1 Ends here %%%%%%%%%%%%%%%%%%%%

\par

\begin{Center}
{\fontsize{14pt}{16.8pt}\selectfont \textbf{EJERCICIOS PROPUESTOS}\par}
\end{Center}\par


\vspace{\baselineskip}

\vspace{\baselineskip}
\setstretch{2.0}
\begin{Center}
{\fontsize{14pt}{16.8pt}\selectfont \textbf{Presenta}\par}
\end{Center}\par

\setstretch{1.0}
\begin{Center}
{\fontsize{14pt}{16.8pt}\selectfont Cristian David Mora Sáenz\par}
\end{Center}\par


\vspace{\baselineskip}

\vspace{\baselineskip}

\vspace{\baselineskip}
\setstretch{2.0}
\begin{Center}
{\fontsize{14pt}{16.8pt}\selectfont \textbf{Docente }\par}
\end{Center}\par

\begin{Center}
{\fontsize{14pt}{16.8pt}\selectfont Segundo Fidel Puerto Garavito\par}
\end{Center}\par


\vspace{\baselineskip}
\setstretch{1.0}

\vspace{\baselineskip}

\vspace{\baselineskip}

\vspace{\baselineskip}
\setstretch{2.0}
\begin{Center}
{\fontsize{14pt}{16.8pt}\selectfont \textbf{Asignatura}\par}
\end{Center}\par

\setstretch{1.0}
\begin{Center}
{\fontsize{14pt}{16.8pt}\selectfont Diseño de algoritmos\par}
\end{Center}\par


\vspace{\baselineskip}

\vspace{\baselineskip}

\vspace{\baselineskip}

\vspace{\baselineskip}
\begin{Center}
{\fontsize{14pt}{16.8pt}\selectfont NRC: 7487\\
\\
\\
\\
\par}
\end{Center}\par


\vspace{\baselineskip}

\vspace{\baselineskip}
\\
\\
{\fontsize{14pt}{16.8pt}\selectfont Bogotá D.C, Colombia\tab \tab \tab \tab \tab Marzo 10 de 2020.\par}\par


\vspace{\baselineskip}

\vspace{\baselineskip}

\vspace{\baselineskip}

\vspace{\baselineskip}

\vspace{\baselineskip}
\textbf{SOLUCIÓN DE EJERCICIOS PROPUESTOS } \par


\vspace{\baselineskip}
\textbf{1.1 }\par

\begin{adjustwidth}{0.58in}{2.66in}
{\fontsize{10pt}{12.0pt}\selectfont \textit{(i) n}\textsuperscript{2}$ \in $ O(\textit{n}\textsuperscript{3}) es cierto pues \textsubscript{$\infty$ $ \rightarrow $ \textit{n }}\textit{lim }(\textit{n}\textsuperscript{2}/\textit{n}\textsuperscript{3}) = 0. \par}\par

\end{adjustwidth}

\begin{adjustwidth}{0.58in}{2.71in}
{\fontsize{10pt}{12.0pt}\selectfont \textit{(ii) n}\textsuperscript{3}$ \in $ O(\textit{n}\textsuperscript{2}) es falso pues \textsubscript{$\infty$ $ \rightarrow $ \textit{n }}\textit{lim }(\textit{n}\textsuperscript{2}/\textit{n}\textsuperscript{3}) = 0. \par}\par

\end{adjustwidth}

\begin{adjustwidth}{0.58in}{2.44in}
{\fontsize{10pt}{12.0pt}\selectfont \textit{(iii) }2\textit{\textsuperscript{n+}}\textsuperscript{1}$ \in $ O(2\textit{\textsuperscript{n}}) es cierto pues \textsubscript{$\infty$ $ \rightarrow $ \textit{n }}\textit{lim }(2\textit{\textsuperscript{n}}\textsuperscript{+1}/2\textit{\textsuperscript{n}}) = 2. \par}\par

\end{adjustwidth}

\begin{adjustwidth}{0.58in}{2.17in}
{\fontsize{10pt}{12.0pt}\selectfont \textit{(iv) }(\textit{n}+1)!$ \in $ O(\textit{n}!) es falso pues \textsubscript{$\infty$ $ \rightarrow $ \textit{n }}\textit{lim }(\textit{n}!/(\textit{n}+1)!) = 0. \par}\par

\end{adjustwidth}

\begin{adjustwidth}{0.58in}{0.57in}
\begin{justify}
{\fontsize{10pt}{12.0pt}\selectfont \textit{(v) f}(\textit{n})$ \in $ O(\textit{n}) $ \Rightarrow $  2\textit{\textsuperscript{f}}\textsuperscript{(\textit{n})}$ \in $ O(2\textit{\textsuperscript{n}}) es falso. Por ejemplo, sea \textit{f}(\textit{n}) = 3\textit{n}; claramente \textit{f}(\textit{n})$ \in $ O(\textit{n}) pero sin embargo \textsubscript{$\infty$ $ \rightarrow $ \textit{n }}\textit{lim }(2\textit{\textsuperscript{n}}/2\textsuperscript{3\textit{n}}) = 0, con lo cual 2\textsuperscript{3\textit{n}}$ \notin $ O(2\textit{\textsuperscript{n}}). De forma más general, resulta ser falso para cualquier función lineal de la forma \textit{f}(\textit{n}) = {\fontsize{11pt}{13.2pt}\selectfont $ \alpha $ {\fontsize{10pt}{12.0pt}\selectfont \textit{n }con {\fontsize{11pt}{13.2pt}\selectfont $ \alpha $  {\fontsize{10pt}{12.0pt}\selectfont > 1, y cierto para \textit{f}(\textit{n}) = {\fontsize{11pt}{13.2pt}\selectfont $ \beta $ {\fontsize{10pt}{12.0pt}\selectfont \textit{n }con {\fontsize{11pt}{13.2pt}\selectfont $ \beta $  {\fontsize{10pt}{12.0pt}\selectfont $ \leq $  1. \par}\par}\par}\par}\par}\par}\par}\par}\par}
\end{justify}\par

\end{adjustwidth}

\begin{adjustwidth}{0.58in}{2.71in}
{\fontsize{10pt}{12.0pt}\selectfont \textit{(vi) }3\textit{\textsuperscript{n}}$ \in $ O(2\textit{\textsuperscript{n}}) es falso pues \textsubscript{$\infty$ $ \rightarrow $ \textit{n }}\textit{lim }(2\textit{\textsuperscript{n}}/3\textit{\textsuperscript{n}}) = 0. \par}\par

\end{adjustwidth}

\begin{adjustwidth}{0.58in}{2.17in}
{\fontsize{10pt}{12.0pt}\selectfont \textit{(vii) }log\textit{n }$ \in $ O(\textit{n}\textsuperscript{1/2}) es cierto pues \textsubscript{$\infty$ $ \rightarrow $ \textit{n }}\textit{lim }(log\textit{n}/\textit{n}\textsuperscript{1/2}) = 0. \par}\par

\end{adjustwidth}

\begin{adjustwidth}{0.58in}{2.26in}
{\fontsize{10pt}{12.0pt}\selectfont \textit{(viii) n}\textsuperscript{1/2}$ \in $ O(log\textit{n}) es falso pues \textsubscript{$\infty$ $ \rightarrow $ \textit{n }}\textit{lim }(log\textit{n}/\textit{n}\textsuperscript{1/2}) = 0. \par}\par

\end{adjustwidth}

\begin{adjustwidth}{0.58in}{2.7in}
{\fontsize{10pt}{12.0pt}\selectfont \textit{(ix) n}\textsuperscript{2}$ \in $ $ \Omega $ (\textit{n}\textsuperscript{3}) es falso pues \textsubscript{$\infty$ $ \rightarrow $ \textit{n }}\textit{lim }(\textit{n}\textsuperscript{2}/\textit{n}\textsuperscript{3}) = 0. \par}\par

\end{adjustwidth}

\begin{adjustwidth}{0.58in}{2.65in}
{\fontsize{10pt}{12.0pt}\selectfont \textit{(x) n}\textsuperscript{3}$ \in $ $ \Omega $ (\textit{n}\textsuperscript{2}) es cierto pues \textsubscript{$\infty$ $ \rightarrow $ \textit{n }}\textit{lim }(\textit{n}\textsuperscript{2}/\textit{n}\textsuperscript{3}) = 0. \par}\par

\end{adjustwidth}

\begin{adjustwidth}{0.58in}{2.43in}
{\fontsize{10pt}{12.0pt}\selectfont \textit{(xi) }2\textit{\textsuperscript{n+}}\textsuperscript{1}$ \in $ $ \Omega $ (2\textit{\textsuperscript{n}}) es cierto pues \textsubscript{$\infty$ $ \rightarrow $ \textit{n }}\textit{lim }(2\textit{\textsuperscript{n}}\textsuperscript{+1}/2\textit{\textsuperscript{n}}) = 2. \par}\par

\end{adjustwidth}

\begin{adjustwidth}{0.58in}{2.12in}
{\fontsize{10pt}{12.0pt}\selectfont \textit{(xii) }(\textit{n}+1)!$ \in $ $ \Omega $ (\textit{n}!) es cierto pues \textsubscript{$\infty$ $ \rightarrow $ \textit{n }}\textit{lim }(\textit{n}!/(\textit{n}+1)!) = 0. \par}\par

\end{adjustwidth}

\begin{adjustwidth}{0.58in}{0.57in}
\begin{justify}
{\fontsize{10pt}{12.0pt}\selectfont \textit{(xiii) f}(\textit{n})$ \in $ $ \Omega $ (\textit{n}) $ \Rightarrow $  2\textit{\textsuperscript{f}}\textsuperscript{(\textit{n})}$ \in $ $ \Omega $ (2\textit{\textsuperscript{n}}) es falso. Por ejemplo, sea \textit{f}(\textit{n}) = (1/2)\textit{n}; claramente \textit{f}(\textit{n})$ \in $ O(\textit{n}) pero sin embargo \textsubscript{$\infty$ $ \rightarrow $ \textit{n }}\textit{lim }(2\textsuperscript{(1/2)\textit{n}}/2\textit{\textsuperscript{n}}) = 0, con lo cual 2\textsuperscript{(1/2)\textit{n}}$ \notin $ $ \Omega $ (2\textit{\textsuperscript{n}}). De forma más general, resulta ser falso para cualquier función \textit{f}(\textit{n}) = {\fontsize{11pt}{13.2pt}\selectfont $ \alpha $ {\fontsize{10pt}{12.0pt}\selectfont \textit{n }con {\fontsize{11pt}{13.2pt}\selectfont $ \alpha $  {\fontsize{10pt}{12.0pt}\selectfont < 1, y cierto para \textit{f}(\textit{n}) = {\fontsize{11pt}{13.2pt}\selectfont $ \beta $ {\fontsize{10pt}{12.0pt}\selectfont \textit{n }con {\fontsize{11pt}{13.2pt}\selectfont $ \beta $  {\fontsize{10pt}{12.0pt}\selectfont $ \geq $  1. \par}\par}\par}\par}\par}\par}\par}\par}\par}
\end{justify}\par

\end{adjustwidth}

\begin{adjustwidth}{0.58in}{2.65in}
{\fontsize{10pt}{12.0pt}\selectfont \textit{(xiv) }3\textit{\textsuperscript{n}}$ \in $ $ \Omega $ (2\textit{\textsuperscript{n}}) es cierto pues \textsubscript{$\infty$ $ \rightarrow $ \textit{n }}\textit{lim }(2\textit{\textsuperscript{n}}/3\textit{\textsuperscript{n}}) = 0. \par}\par

\end{adjustwidth}

\begin{adjustwidth}{0.58in}{2.65in}
{\fontsize{10pt}{12.0pt}\selectfont \textit{(xv) }log\textit{n }$ \in $ $ \Omega $ (\textit{n}\textsuperscript{1/2}) es falso pues \textsubscript{$\infty$ $ \rightarrow $ \textit{n }}\textit{lim }(log\textit{n}/\textit{n}\textsuperscript{1/2}) =0. \par}\par

\end{adjustwidth}

\begin{adjustwidth}{0.58in}{2.2in}
{\fontsize{10pt}{12.0pt}\selectfont \textit{(xvi) n}\textsuperscript{1/2}$ \in $ $ \Omega $ (log\textit{n}) es cierto pues \textsubscript{$\infty$ $ \rightarrow $ \textit{n }}\textit{lim }(log\textit{n}/\textit{n}\textsuperscript{1/2}) = 0. \par}\par

\end{adjustwidth}


\vspace{\baselineskip}
\begin{adjustwidth}{0.58in}{0.57in}
{\fontsize{10pt}{12.0pt}\selectfont \textbf{Solución al Problema 1.2 }(☺) \par}\par

\end{adjustwidth}

\begin{adjustwidth}{0.58in}{2.42in}
{\fontsize{10pt}{12.0pt}\selectfont $\bullet$  Respecto al orden de complejidad O tenemos que: \par}\par

\end{adjustwidth}

\begin{adjustwidth}{0.91in}{0.71in}
\begin{Center}
{\fontsize{10pt}{12.0pt}\selectfont O(\textit{n}log\textit{n}) $ \subset $  O(\textit{n}\textsuperscript{1+\textit{a}}) $ \subset $  O(\textit{n}\textsuperscript{2}/log\textit{n}) $ \subset $  O(\textit{n}\textsuperscript{2}log\textit{n}) $ \subset $  O(\textit{n}\textsuperscript{8}) = O((\textit{n}\textsuperscript{2}+8\textit{n}+log\textsuperscript{3}\textit{n})\textsuperscript{4}) $ \subset $  O((1+\textit{a})\textit{\textsuperscript{n}}) $ \subset $  O(2\textit{\textsuperscript{n}}). \par}
\end{Center}\par

\end{adjustwidth}

\begin{adjustwidth}{0.77in}{0.57in}
\begin{justify}
{\fontsize{10pt}{12.0pt}\selectfont Puesto que todas las funciones son continuas, para comprobar que O(\textit{f})$ \subset $ O(\textit{g}), basta ver que \textsubscript{$\infty$ $ \rightarrow $ \textit{n }}\textit{lim }(\textit{f}(\textit{n})/\textit{g}(\textit{n})) = 0, y para comprobar que O(\textit{f}) = O(\textit{g}), basta ver que \textsubscript{$\infty$ $ \rightarrow $ \textit{n }}\textit{lim }(\textit{f}(\textit{n})/\textit{g}(\textit{n})) es finito y distinto de 0. \par}
\end{justify}\par

\end{adjustwidth}

\begin{adjustwidth}{0.58in}{1.4in}
{\fontsize{10pt}{12.0pt}\selectfont $\bullet$  Por otro lado, respecto al orden de complejidad $ \Omega $ , obtenemos que: \par}\par

\end{adjustwidth}

\begin{adjustwidth}{0.82in}{0.61in}
\begin{Center}
{\fontsize{10pt}{12.0pt}\selectfont $ \Omega $ (\textit{n}log\textit{n}) $\supset$  $ \Omega $ (\textit{n}\textsuperscript{1+\textit{a}}) $\supset$  $ \Omega $ (\textit{n}\textsuperscript{2}/log\textit{n}) $\supset$  $ \Omega $ (\textit{n}\textsuperscript{2}log\textit{n}) $\supset$  $ \Omega $ (\textit{n}\textsuperscript{8}) = $ \Omega $ ((\textit{n}\textsuperscript{2}+8\textit{n}+log\textsuperscript{3}\textit{n})\textsuperscript{4}) $\supset$  $ \Omega $ ((1+\textit{a})\textit{\textsuperscript{n}}) $\supset$  $ \Omega $ (2\textit{\textsuperscript{n}}) \par}
\end{Center}\par

\end{adjustwidth}

\begin{adjustwidth}{0.77in}{0.57in}
\begin{justify}
{\fontsize{10pt}{12.0pt}\selectfont Para comprobar que $ \Omega $ (\textit{f}) $ \subset $  $ \Omega $ (\textit{g}), basta ver que \textsubscript{$\infty$ $ \rightarrow $ \textit{n }}\textit{lim }(\textit{g}(\textit{n})/\textit{f}(\textit{n})) = 0, y para comprobar que $ \Omega $ (\textit{f}) = $ \Omega $ (\textit{g}), basta ver que \textsubscript{$\infty$ $ \rightarrow $ \textit{n }}\textit{lim }(\textit{f}(\textit{n})/\textit{g}(\textit{n})) es finito y distinto de 0 puesto que al ser las funciones continuas tenemos garantizada la existencia de los límites. \par}
\end{justify}\par

\end{adjustwidth}

\begin{adjustwidth}{0.58in}{0.57in}
{\fontsize{10pt}{12.0pt}\selectfont $\bullet$  Y en lo relativo al orden de complejidad $ \Theta $ , al definirse como la intersección de \textsubscript{los órdenes O y $ \Omega $ , sólo tenemos asegurado que: }\par}\par

\end{adjustwidth}

\begin{adjustwidth}{2.3in}{2.29in}
{\fontsize{10pt}{12.0pt}\selectfont $ \Theta $ (\textit{n}\textsuperscript{8}) = $ \Theta $ ((\textit{n}\textsuperscript{2}+8\textit{n}+log\textsuperscript{3}\textit{n})\textsuperscript{4}), \par}\par

\end{adjustwidth}

\begin{adjustwidth}{0.77in}{0.96in}
{\fontsize{10pt}{12.0pt}\selectfont siendo los órdenes $ \Theta $  del resto de las funciones conjuntos no comparables. \par}\par

\end{adjustwidth}

\begin{Center}
{\fontsize{10pt}{12.0pt}\selectfont Igualando ahora los coeficientes que acompañan a \textit{n\textsuperscript{k }}obtenemos que \textit{\textsubscript{c}}\textsubscript{que \textit{k}+1 – \textit{a }< \textit{ac}1 \textit{k}+1 y = \textit{c }> \textit{c}, 0, o entonces lo que es \textit{c}igual, \textit{k}+1 no puede (1–\textit{a})\textit{c}ser \textit{k}+1 = cero. \textit{c}. Ahora bien, como sabemos }– Si \textit{a }> 1, las funciones del segundo sumatorio son exponenciales, mientras \textsubscript{que este las caso primeras el orden se de mantienen complejidad dentro del de algoritmo un orden es polinomial, exponencial. por Ahora lo que bien, en como todas las raíces \textit{b}-ésimas de \textit{a }tienen el mismo módulo, todas crecen de la misma forma y por tanto todas son del mismo orden de complejidad, obteniendo }\par}
\end{Center}\par

{\fontsize{7pt}{8.4pt}\selectfont \textit{n }{\fontsize{10pt}{12.0pt}\selectfont de cero y finito, podemos concluir que: \par}\par}\par

\begin{justify}
{\fontsize{10pt}{12.0pt}\selectfont Hemos \textsubscript{para todas las }supuesto \textsubscript{condiciones }que \textit{d}\textsubscript{1 iniciales, $ \neq $  0. Esto no aunque tiene por sin qué ser necesariamente embargo sí cierto es cierto que al menos uno de los sumando para demostrar coeficientes lo anterior. \textit{di }ha de ser distinto de cero. Basta tomar ese }{\fontsize{6pt}{7.2pt}\selectfont † \textsubscript{Recordemos que dados dos números reales \textit{a }y \textit{b}, la solución de la ecuación \textit{x}}\textit{b }\textsubscript{– \textit{a }= 0 }{\fontsize{10pt}{12.0pt}\selectfont tiene \textit{b }raíces distintas, que pueden ser expresadas como \textit{a}\textsuperscript{1/\textit{b}}e\textsuperscript{2$ \pi $ \textit{ik}/\textit{n}}, para \textit{k}=0,1,2,...,\textit{n}–1. \par}\par}\par}
\end{justify}\par

{\fontsize{7pt}{8.4pt}\selectfont LA COMPLEJIDAD DE LOS ALGORITMOS {\fontsize{10pt}{12.0pt}\selectfont 25 \par}\par}\par

{\fontsize{10pt}{12.0pt}\selectfont $\bullet$  Supongamos ahora que \textit{a }= 1. En este caso la multiplicidad de la raíz 1 es \textit{k}+2, \textsubscript{con lo cual }\par}\par

\textsubscript{tanto }{\fontsize{10pt}{12.0pt}\selectfont Pero \textsubscript{el segundo }las raíces \textsubscript{sumando }\textit{r}\textsubscript{2,\textit{r}3,...,\textit{r}de \textit{b }son \textit{T}(\textit{n}) todas es de de complejidad módulo 1 (obsérvese $ \Theta $ (1). que \textit{r}1=1), y por polinomio }Así, el crecimiento \textsubscript{de grado \textit{k}+1 }de \textsubscript{con }\textit{T}(\textit{n}) \textsubscript{lo cual }coincide \textit{\textsubscript{T}}\textsubscript{(\textit{n})$ \in $ $ \Theta $ (\textit{n}}con el {\fontsize{7pt}{8.4pt}\selectfont \textit{k}+1{\fontsize{10pt}{12.0pt}\selectfont del \textsubscript{). }primer sumando, que es un \textbf{Solución al Problema 1.4 }( ) \par}\par}\par}\par

{\fontsize{10pt}{12.0pt}\selectfont Haciendo el cambio \textit{n }= \textit{b\textsuperscript{m}}, o lo que es igual, \textit{m }= log\textit{\textsubscript{bn}}\textsubscript{, obtenemos que }\par}\par

{\fontsize{10pt}{12.0pt}\selectfont \textit{T}(\textit{b\textsuperscript{m}}) = \textit{aT}(\textit{b\textsuperscript{m}}\textsuperscript{–1}) + \textit{cb\textsuperscript{mk}}. Llamando \textit{t\textsubscript{m }}\textsubscript{= \textit{T}(\textit{b}}{\fontsize{7pt}{8.4pt}\selectfont \textit{m}\textsubscript{), la ecuación queda como }\par}\par}\par

{\fontsize{10pt}{12.0pt}\selectfont \textit{t\textsubscript{m }}\textsubscript{– \textit{atk}–1 = \textit{c}(\textit{b}}{\fontsize{7pt}{8.4pt}\selectfont \textit{k}\textsubscript{)}\textit{m}\textsubscript{, }{\fontsize{10pt}{12.0pt}\selectfont ecuación en recurrencia no homogénea con ecuación característica (\textit{x}–\textit{a})(\textit{x}–\textit{b\textsuperscript{k}}) = 0. Para \textsubscript{ecuación }resolver esta \textsubscript{característica es }ecuación, \textsubscript{(\textit{x}–\textit{b}}{\fontsize{7pt}{8.4pt}\selectfont \textit{k}\textsubscript{)}2 \textsubscript{= }{\fontsize{10pt}{12.0pt}\selectfont supongamos \textsubscript{0 y por tanto }\par}\par}\par}\par}\par}\par

{\fontsize{10pt}{12.0pt}\selectfont primero que \textit{a }= \textit{b\textsuperscript{k}}. Entonces, la \textit{t\textsubscript{m }}\textsubscript{= \textit{c}1\textit{b}}{\fontsize{7pt}{8.4pt}\selectfont \textit{km }\textsubscript{+ \textit{c}2\textit{mb}}\textit{km}\textsubscript{. }{\fontsize{10pt}{12.0pt}\selectfont Necesitamos ahora deshacer los cambios hechos. Primero \textit{t\textsubscript{m }}\textsubscript{= \textit{T}(\textit{b}}{\fontsize{7pt}{8.4pt}\selectfont \textit{m}\textsubscript{) con lo que }\par}\par}\par}\par}\par

{\fontsize{10pt}{12.0pt}\selectfont \textit{T}(\textit{b\textsuperscript{m}}) = \textit{c}\textsubscript{1\textit{b}}{\fontsize{7pt}{8.4pt}\selectfont \textit{km }\textsubscript{+ \textit{c}2\textit{mb}}\textit{km }\textsubscript{= (\textit{c}1 + \textit{c}2\textit{m})\textit{b}}\textit{km\textsubscript{, }}{\fontsize{10pt}{12.0pt}\selectfont y después \textit{n }= \textit{b\textsuperscript{m}}, obteniendo finalmente que \par}\par}\par}\par

{\fontsize{10pt}{12.0pt}\selectfont \textit{T}(\textit{n}) = (\textit{c}\textsubscript{1 + \textit{c}2log\textit{bn})\textit{n}}{\fontsize{7pt}{8.4pt}\selectfont \textit{k }\textsubscript{$ \in $  $ \Theta $ (\textit{n}}\textit{k}\textsubscript{log\textit{n}).}$\ddag$  \textsubscript{tiene }{\fontsize{10pt}{12.0pt}\selectfont Supongamos \textsubscript{dos raíces distintas, }ahora el \textsubscript{y }caso \textsubscript{por }contrario, \textsubscript{tanto }\par}\par}\par}\par

{\fontsize{10pt}{12.0pt}\selectfont \textit{a }$ \neq $  \textit{b\textsuperscript{k}}. Entonces la ecuación característica \textit{t\textsubscript{m }}\textsubscript{= \textit{c}1\textit{a}}{\fontsize{7pt}{8.4pt}\selectfont \textit{m }\textsubscript{+ \textit{c}2\textit{b}}\textit{km}\textsubscript{. }{\fontsize{10pt}{12.0pt}\selectfont Necesitamos deshacer los cambios hechos. Primero \textit{t\textsubscript{m }}\textsubscript{= \textit{T}(\textit{b}}{\fontsize{7pt}{8.4pt}\selectfont \textit{m}\textsubscript{), con lo que }\par}\par}\par}\par}\par

{\fontsize{10pt}{12.0pt}\selectfont \textit{T}(\textit{b\textsuperscript{m}}) = \textit{c}\textsubscript{1\textit{a}}{\fontsize{7pt}{8.4pt}\selectfont \textit{m }\textsubscript{+ \textit{c}2\textit{b}}\textit{km}\textsubscript{, }{\fontsize{10pt}{12.0pt}\selectfont y después \textit{n = b\textsuperscript{m}}, obteniendo finalmente que \par}\par}\par}\par

\textit{T}(\textit{n}) = \textit{c}\textsubscript{1}\textit{a}\textsuperscript{log}{\fontsize{5pt}{6.0pt}\selectfont \textit{b \textsuperscript{n }}+ \textit{c}\textsubscript{2}\textit{n \textsuperscript{k }}= \textit{c}\textsubscript{1}\textit{n}\textsuperscript{log }\textit{b \textsuperscript{a }}+ \textit{c}\textsubscript{2}\textit{n \textsuperscript{k}}{\fontsize{10pt}{12.0pt}\selectfont . \par}\par}\par

{\fontsize{6pt}{7.2pt}\selectfont $\ddag$  \textsubscript{Obsérvese que se }{\fontsize{10pt}{12.0pt}\selectfont 1.7. \par}\par}\par

\textsubscript{hace uso de que log\textit{bn}$ \in $ $ \Theta $ (log\textit{n}), lo que se demuestra en el problema }\par

{\fontsize{10pt}{12.0pt}\selectfont 26 {\fontsize{7pt}{8.4pt}\selectfont TÉCNICAS DE DISEÑO DE ALGORITMOS \par}\par}\par

\textsubscript{no, }{\fontsize{10pt}{12.0pt}\selectfont En \textsubscript{es decir, }consecuencia, \textit{\textsubscript{a }}\textsubscript{< \textit{b}}{\fontsize{7pt}{8.4pt}\selectfont \textit{k}\textsubscript{, entonces }{\fontsize{10pt}{12.0pt}\selectfont si log\textit{\textsubscript{ba T}}\textsubscript{(\textit{n}) > \textit{k }(si $ \in $ $ \Theta $ (\textit{n}y sólo }{\fontsize{7pt}{8.4pt}\selectfont \textit{k}\textsubscript{). si \textit{a }> \textit{b}}\textit{k}\textsubscript{) entonces \textit{T}(\textit{n}) $ \in $ $ \Theta $ (\textit{n}}log\textit{\textsubscript{b }a}\textsubscript{). Si }{\fontsize{10pt}{12.0pt}\selectfont \textbf{Solución al Problema 1.5 }\par}\par}\par}\par}\par}\par

\begin{justify}
{\fontsize{10pt}{12.0pt}\selectfont \textbf{Procedimiento \textit{Algoritmo1 }}(☺) a) Para obtener el tiempo de ejecución, calcularemos primero el número de \textsubscript{operaciones elementales (OE) que se realizan: }– En la línea (1) se ejecutan 3 OE (una asignación, una resta y una comparación) \textsubscript{en cada una de las iteraciones del bucle más otras 3 al final, cuando se efectúa la salida del \textit{FOR}. }– Igual ocurre con la línea (2), también con 3 OE (una asignación, una suma y una \textsubscript{comparación) por iteración, más otras 3 al final del bucle. }– En la línea (3) se efectúa una condición, con un total de 4 OE (una diferencia, \textsubscript{dos accesos a un vector, y una comparación). }– Las líneas (4) a (6) sólo se ejecutan si se cumple la condición de la línea (3), y \textsubscript{realizan un total de 9 OE: 3, 4 y 2 respectivamente. }\par}
\end{justify}\par

{\fontsize{10pt}{12.0pt}\selectfont Con esto: \par}\par

\begin{justify}
{\fontsize{10pt}{12.0pt}\selectfont b) Como los tiempos de ejecución en los tres casos son polinomios de grado 2, la \textsubscript{complejidad del algoritmo es cuadrática, independientemente de qué caso se trate. }Obsérvese cómo hemos analizado el tiempo de ejecución del algoritmo sólo en \textsubscript{función de su código y no respecto a lo que hace, puesto que en muchos casos esto nos objetivo llevaría para a el conclusiones que fue diseñado erróneas. el algoritmo. Debe ser a posteriori cuando se analice el procedimiento }En el caso \textsubscript{nos }que \textsubscript{muestra }nos ocupa, \textsubscript{que el algoritmo }un examen \textsubscript{está }más \textsubscript{diseñado }detallado \textsubscript{para ordenar }del código \textsubscript{de forma }del \textsubscript{creciente el vector que se le pasa como parámetro, siguiendo el método de la Burbuja. Lo que acabamos de ver es que sus casos mejor, peor y medio se producen creciente, decreciente respectivamente y aleatoria. cuando el vector está inicialmente ordenado de forma }\textbf{Función \textit{Algoritmo2 }}( ) \par}
\end{justify}\par

{\fontsize{10pt}{12.0pt}\selectfont a) Para calcular el tiempo de ejecución, calcularemos primero el número de \textsubscript{operaciones elementales (OE) que se realizan: }\par}\par

{\fontsize{10pt}{12.0pt}\selectfont – En la línea (1) se ejecutan 2 OE (dos asignaciones). – En la línea (2) se efectúa la condición del bucle, que supone 1 OE (la \textsubscript{comparación). }– Las líneas (3) a (6) componen el cuerpo del bucle, y contabilizan 3, 2+1, 2+2 y \textsubscript{2 finalizar OE respectivamente. si se verifica la Es condición importante de la hacer línea notar (4). que el bucle también puede }– Por \textsubscript{bucle }último, \textit{\textsubscript{WHILE }}la \textsubscript{deja }línea \textsubscript{de }(9) \textsubscript{ser }supone \textsubscript{cierta. }1 OE. A ella se llega cuando la condición del Con esto: \par}\par

{\fontsize{10pt}{12.0pt}\selectfont 28 {\fontsize{7pt}{8.4pt}\selectfont TÉCNICAS DE DISEÑO DE ALGORITMOS \par}\par}\par

{\fontsize{10pt}{12.0pt}\selectfont $\bullet$  En el \textit{caso mejor }se efectuarán solamente la líneas (1), (2), (3) y (4). En \textsubscript{consecuencia, \textit{T}(\textit{n}) = 2+1+3+3 = 9. }\par}\par

{\fontsize{10pt}{12.0pt}\selectfont $\bullet$  En el \textit{caso peor }se efectúa la línea (1), y después se repite el bucle hasta que su \textsubscript{condición iteración del sea bucle falsa, está acabando compuesta la función por las al líneas ejecutarse (2) a (8), la línea junto (9). con Cada una ejecución adicional de la línea (2) que es la que ocasiona la salida del bucle. En cada iteración se reducen a la mitad los elementos a considerar, por lo que el bucle se repite log\textit{n }veces. Por tanto, }\par}\par

{\fontsize{10pt}{12.0pt}\selectfont \textit{T}(\textit{n}) = 2 \par}\par

\textsubscript{+ }\textsuperscript{⎛ │}│⎝{\fontsize{7pt}{8.4pt}\selectfont loglog1011)22231( \textsubscript{1 }4 ⎛ │\textsubscript{│⎝$ \sum $ = ++++ }⎞ │\textsubscript{│⎠}+ ⎞ \par}\par

│\textsubscript{│⎠}{\fontsize{7pt}{8.4pt}\selectfont \textit{n }\par}\par

=+ \textit{n }+ {\fontsize{7pt}{8.4pt}\selectfont \textit{i }\textsuperscript{. }\par}\par

\begin{justify}
{\fontsize{10pt}{12.0pt}\selectfont $\bullet$  En \textsubscript{el bucle, }el \textit{caso }\textsubscript{y }\textit{medio}, \textsubscript{para esto }necesitamos \textsubscript{veamos cuántas }calcular \textsubscript{veces }el número \textsubscript{puede }medio \textsubscript{repetirse, }de veces \textsubscript{y qué }que \textsubscript{probabilidad }se repite \textsubscript{tiene cada una de suceder. }Por un lado, el bucle puede repetirse desde una vez hasta log\textit{n }veces, puesto \textsubscript{que en cada iteración se divide por dos el número de elementos considerados. Si se repitiese una sola vez, es que el elemento ocuparía la posición \textit{n}/2, lo que ocurre elemento con ocuparía una probabilidad alguna de 1/(\textit{n}+1). las posiciones Si el bucle \textit{n}/4 se ó repitiese 3\textit{n}/4, lo dos cual veces ocurre es que con el probabilidad 1/(\textit{n}+1)+1/(\textit{n}+1)=2/(\textit{n}+1). En el elemento ocuparía alguna de las posiciones general, \textit{nk}/2}{\fontsize{7pt}{8.4pt}\selectfont \textit{i\textsubscript{, }}\textsubscript{si se repitiese con \textit{k }impar y \textit{i }1$ \leq $ \textit{k}<2veces }\textit{i}\textsubscript{. es que }{\fontsize{10pt}{12.0pt}\selectfont Es decir, el bucle se repite \textit{i }veces con probabilidad 2\textit{\textsuperscript{i}}\textsuperscript{–1}/(\textit{n}+1). Por tanto, el \textsubscript{número medio de veces que se repite el ciclo vendrá dado por la expresión: }\par}\par}\par}
\end{justify}\par

{\fontsize{6pt}{7.2pt}\selectfont log{\fontsize{17pt}{20.4pt}\selectfont $ \sum $ {\fontsize{6pt}{7.2pt}\selectfont $-$  \textit{i }\par}\par}\par}\par

\textsubscript{= }\par

{\fontsize{6pt}{7.2pt}\selectfont \textit{n }\par}\par

\textit{\textsuperscript{i }}\par

\textsubscript{1 }\par

{\fontsize{11pt}{13.2pt}\selectfont 2 {\fontsize{6pt}{7.2pt}\selectfont \textit{i }\par}\par}\par

{\fontsize{6pt}{7.2pt}\selectfont 1 \par}\par

{\fontsize{11pt}{13.2pt}\selectfont \textit{n }\textsubscript{+ }\par}\par

{\fontsize{11pt}{13.2pt}\selectfont 1 \par}\par

\textsubscript{= }{\fontsize{11pt}{13.2pt}\selectfont \textit{nnn }\par}\par

{\fontsize{11pt}{13.2pt}\selectfont log \par}\par

{\fontsize{11pt}{13.2pt}\selectfont +$-$  \textit{n }\par}\par

{\fontsize{11pt}{13.2pt}\selectfont + \par}\par

{\fontsize{11pt}{13.2pt}\selectfont 1 \par}\par

{\fontsize{11pt}{13.2pt}\selectfont 1 \textsubscript{. }\par}\par

{\fontsize{10pt}{12.0pt}\selectfont Con esto, la función ejecuta la línea (1) y después el bucle se repite ese \textsubscript{número Por consiguiente, medio de veces, saliendo por la instrucción \textit{RETURN }en la línea (4). }\textit{T}(\textit{n})= \textsubscript{2 + }⎛ \par}\par

\textsubscript{│⎝}\textit{nnn }\par

\begin{Center}
log +$-$  \textit{n }+ 
\end{Center}\par

\textsubscript{1 }\par

1 ⎞ \textsubscript{│⎠89)331()2231( +=++++++ }\textit{nnn }\par

\begin{Center}
log +$-$  \textit{\textsubscript{n }}\textsubscript{+ }
\end{Center}\par

\begin{FlushRight}
\textsubscript{1 }1 {\fontsize{10pt}{12.0pt}\selectfont c) En \textsubscript{medio, }el caso \textsubscript{la complejidad }mejor el tiempo \textsubscript{resultante }de ejecución \textsubscript{es de orden }es una \textsubscript{$ \Theta $ (log\textit{n}) }constante. \textsubscript{puesto }Para \textsubscript{que }los casos peor y \textit{lim \textsubscript{n }}{\fontsize{6pt}{7.2pt}\selectfont $\infty$ $ \rightarrow $  \par}\par}
\end{FlushRight}\par

\begin{Center}
\textit{\textsuperscript{nT }})( \textsubscript{log }\textit{n}{\fontsize{10pt}{12.0pt}\selectfont es una constante finita y distinta de cero en ambos casos (10 y 8 \textsubscript{respectivamente). }\par}
\end{Center}\par

\begin{adjustwidth}{0.58in}{0.57in}
{\fontsize{7pt}{8.4pt}\selectfont LA COMPLEJIDAD DE LOS ALGORITMOS {\fontsize{10pt}{12.0pt}\selectfont 29 \par}\par}\par

\end{adjustwidth}

\begin{adjustwidth}{0.58in}{0.57in}
{\fontsize{10pt}{12.0pt}\selectfont \textbf{Función \textit{Euclides }}( ) \par}\par

\end{adjustwidth}

\begin{adjustwidth}{0.57in}{0.56in}
{\fontsize{10pt}{12.0pt}\selectfont a) En este caso el análisis del tiempo de ejecución y la complejidad de la función \textsubscript{sigue un proceso distinto al estudiado en los casos anteriores. }Lo primero es resaltar algunas características del algoritmo, siguiendo una línea \textsubscript{de razonamiento similar a la de [BRA97]: }[1] Para cualquier par de enteros no negativos \textit{m }y \textit{n }tales que \textit{n}$ \geq $ \textit{m}, se verifica que \textit{\textsubscript{n }}\textsubscript{MOD \textit{m }< \textit{n}/2. Veámoslo: }a) Si \textit{m }> \textit{n}/2 entonces 1$ \leq $  \textit{n}/\textit{m }< 2 y por tanto \textit{n }DIV \textit{m }= 1, lo que implica \textsubscript{que \textit{n }MOD \textit{m }= \textit{n }– \textit{m}(\textit{n }DIV \textit{m}) = \textit{n }– \textit{m }< \textit{n }– \textit{n}/2 = \textit{n}/2. }b) Por otro lado, si \textit{m }$ \leq $  \textit{n}/2 entonces \textit{n }MOD \textit{m }< \textit{m }$ \leq $  \textit{n}/2. [2] Podemos suponer sin pérdida de generalidad que \textit{n }$ \geq $  \textit{m}. Si no, la primera \textsubscript{iteración del bucle intercambia \textit{n }con \textit{m }ya que \textit{n }MOD \textit{m }= \textit{n }cuando \textit{n }< \textit{m}. Además, la condición \textit{n }$ \geq $  \textit{m }se conserva siempre (es decir, es un invariante del bucle) pues \textit{n }MOD \textit{m }nunca es mayor que \textit{m}. }[3] El cuerpo del bucle efectúa 4 OE, con lo cual el tiempo del algoritmo es del \textsubscript{orden exacto del número de iteraciones que realiza el bucle. Por consiguiente, para determinar la complejidad del algoritmo es suficiente acotar este número. }[4] Una propiedad curiosa de este algoritmo es que no se produce un avance \textsubscript{notable con cada iteración del bucle, sino que esto ocurre cada dos iteraciones. Consideremos lo que les ocurre a \textit{m }y \textit{n }cuando el ciclo se repite dos veces, suponiendo que no acaba antes. Sean \textit{m}0 y \textit{n}0 los valores originales de los parámetros, que podemos suponer \textit{n}0 $ \geq $  \textit{m}0 por [2]. Después de la primera iteración, \textit{m }vale \textit{n}0 MOD \textit{m}0. Después de la segunda iteración, \textit{n }toma ese valor, y por tanto ya es menor que \textit{n}0/2 (por [1]). En consecuencia, \textit{n }vale menos de la mitad de lo que valía tras dos iteraciones del bucle. Como se sigue manteniendo que \textit{n }$ \geq $  \textit{m}, el mismo razonamiento se puede repetir para las siguientes dos iteraciones, y así sucesivamente. }\par}\par

\end{adjustwidth}

\begin{adjustwidth}{0.58in}{0.57in}
{\fontsize{10pt}{12.0pt}\selectfont El hecho de que \textit{n }valga menos de la mitad cada dos iteraciones del bucle es el \textsubscript{que nos permite intuir que el bucle se va a repetir del orden de 2log\textit{n }veces. Vamos a demostrar esto formalmente. }Para ello, vamos a tratar el bucle como si fuera un algoritmo recursivo. Sea \textit{T}(\textit{l}) \textsubscript{el número máximo de veces que se repite el bucle para valores iniciales \textit{m }y \textit{n }cuando \textit{m }$ \leq $  \textit{n }$ \leq $  \textit{l}. En este caso \textit{l }representa el tamaño de la entrada. }\par}\par

\end{adjustwidth}

\begin{adjustwidth}{0.57in}{0.57in}
{\fontsize{10pt}{12.0pt}\selectfont – Si \textit{n }$ \leq $  2 el bucle no se repite (si \textit{m }= 0) o se hace una sola vez (si \textit{m }es 1 ó 2). – Si \textit{n }> 2 y \textit{m}=1 o bien \textit{m }divide a \textit{n}, el bucle se repite una sola vez. – En otro caso (\textit{n }> 2 y \textit{m }no divide a \textit{n}) el bucle se ejecuta dos veces, y por lo \textsubscript{visto en [4], \textit{n }vale a lo sumo la mitad de lo que valía inicialmente. En consecuencia \textit{n }$ \leq $  (\textit{l}/2), y además \textit{m }se sigue manteniendo por debajo de \textit{n}. }\par}\par

\end{adjustwidth}

\begin{adjustwidth}{0.57in}{0.57in}
{\fontsize{10pt}{12.0pt}\selectfont Esto nos lleva a la ecuación en recurrencia \textit{T}(\textit{l}) $ \leq $  2 + \textit{T}(\textit{l}/2) si \textit{l }> 2, \textit{T}(\textit{l}) $ \leq $  1 si \textit{l}$ \leq $  \textsubscript{2, lo que implica que el algoritmo de Euclides es de complejidad logarítmica respecto al tamaño de la entrada (\textit{l}). }\par}\par

\end{adjustwidth}

\begin{adjustwidth}{0.58in}{0.57in}
{\fontsize{10pt}{12.0pt}\selectfont 30 {\fontsize{7pt}{8.4pt}\selectfont TÉCNICAS DE DISEÑO DE ALGORITMOS \par}\par}\par

\end{adjustwidth}

\begin{adjustwidth}{0.57in}{0.57in}
\begin{justify}
{\fontsize{10pt}{12.0pt}\selectfont Nos preguntaremos la razón de usar \textit{T}(\textit{l}) para acotar el número de iteraciones \textsubscript{que realiza el algoritmo en vez de definir \textit{T }directamente como una función de \textit{n}, el mayor de los dos operandos, lo cual sería mucho más intuitivo. }El problema es que si definimos \textit{T}(\textit{n}) como el número de iteraciones que realiza \textsubscript{el algoritmo para los valores \textit{m }$ \leq $  \textit{n}, no podríamos concluir que \textit{T}(\textit{n}) $ \leq $  2 + \textit{T}(\textit{n}/2) del hecho de que \textit{n }valga la mitad de su valor tras cada dos iteraciones del bucle. }Por ejemplo, para \textit{Euclides(}8\textit{,}13\textit{)}, obtenemos que \textit{T}(13) = 5 en el peor caso, \textsubscript{mientras que \textit{T}(13/2) = \textit{T}(6) = 2. Esto ocurre porque tras dos iteraciones del bucle \textit{n }no vale 6, sino 5 (y \textit{m }= 3), y con esto sí es cierto que \textit{T}(13) $ \leq $  2 + \textit{T}(5) ya que \textit{T}(5) = 3. }La raíz de este problema es que esta nueva definición más intuitiva de \textit{T }no \textsubscript{lleva a una función monótona no decreciente (\textit{T}(5) > \textit{T}(6)) y por tanto la existencia de algún \textit{n}’ $ \leq $  \textit{n}/2 tal que \textit{T}(\textit{n}) $ \leq $  2 + \textit{T}(\textit{n}’) no implica necesariamente que \textit{T}(\textit{n}) $ \leq $  2 + \textit{T}(\textit{n}/2). }En vez de esto, solamente podríamos afirmar que \textit{T}(\textit{n})$ \leq $ 2+\textit{max}$ \{ $ \textit{T}(\textit{n}’)$ \vert $ \textit{n}’$ \leq $ \textit{n}/2$ \} $ , \textsubscript{que es una ecuación en recurrencia bastante difícil de resolver. Esa es la razón de que escogiésemos nuestra función \textit{T }de forma que fuera no decreciente y que expresara una cota superior del número de iteraciones. }Para acabar, es interesante hacer notar una característica curiosa de este \textsubscript{algoritmo: se demuestra que su caso peor ocurre cuando \textit{m }y \textit{n }son dos términos consecutivos de la sucesión de Fibonacci. }\par}
\end{justify}\par

\end{adjustwidth}

\begin{adjustwidth}{0.58in}{0.57in}
{\fontsize{10pt}{12.0pt}\selectfont b) \textit{T}(\textit{l})$ \in $ $ \Theta $ (log\textit{l}) como se deduce de la ecuación en recurencia que define el tiempo \textsubscript{de ejecución del algoritmo. }\par}\par

\end{adjustwidth}

\begin{adjustwidth}{0.58in}{0.57in}
\begin{justify}
{\fontsize{10pt}{12.0pt}\selectfont \textbf{Procedimiento \textit{Misterio }}(☺) a) En este caso son tres bucles anidados los que se ejecutan, independientemente de \textsubscript{los valores de la entrada, es decir, no existe peor, medio o mejor caso, sino un único caso. }Para calcular el tiempo de ejecución, veamos el número de operaciones \textsubscript{elementales (OE) que se realizan: }\par}
\end{justify}\par

\end{adjustwidth}

\begin{adjustwidth}{0.58in}{0.57in}
{\fontsize{10pt}{12.0pt}\selectfont – En la línea (1) se ejecuta 1 OE (una asignación). – En la línea (2) se ejecutarán 3 OE (una asignación, una resta y una \textsubscript{comparación) en cada una de las iteraciones del bucle más otras 3 al final, cuando se efectúa la salida del \textit{FOR}. }– Igual ocurre con la línea (3), también con 3 OE (una asignación, una suma y una \textsubscript{comparación) por iteración, más otras 3 al final del bucle. }– Y también en la línea (4), esta vez con 2 OE (asignación y comparación) más \textsubscript{las 2 adicionales de terminación del bucle. }– Por último, la línea (5) supone 2 OE (un incremento y una asignación). \par}\par

\end{adjustwidth}

\begin{adjustwidth}{0.58in}{0.57in}
{\fontsize{10pt}{12.0pt}\selectfont Con esto, el bucle interno se ejecutará \textit{j }veces, el medio (\textit{n}–\textit{i}) veces, y el bucle \textsubscript{exterior (\textit{n}–1) veces, lo que conlleva un tiempo de ejecución de: }\par}\par

\end{adjustwidth}

{\fontsize{7pt}{8.4pt}\selectfont LA COMPLEJIDAD DE LOS ALGORITMOS {\fontsize{10pt}{12.0pt}\selectfont 31 \par}\par}\par

{\fontsize{10pt}{12.0pt}\selectfont b) Como \textsubscript{algoritmo }el \textsubscript{es }tiempo \textsubscript{de orden }de \textsubscript{$ \Theta $ (\textit{n}}ejecución {\fontsize{7pt}{8.4pt}\selectfont 3\textsubscript{). }{\fontsize{10pt}{12.0pt}\selectfont es un polinomio de grado 3, la complejidad del \textbf{Solución al Problema 1.6 }(☺) \par}\par}\par}\par

\begin{justify}
{\fontsize{10pt}{12.0pt}\selectfont Para comprobar que O(\textit{f}) $ \subset $  O(\textit{g}) en cada caso y que esa inclusión es estricta, basta ver que \textit{lim }\textsubscript{$\infty$ $ \rightarrow $ \textit{n }}(\textit{f}(\textit{n})/\textit{g}(\textit{n})) = 0, pues todas las funciones son continuas y por tanto los límites existen. Por consiguiente, \par}
\end{justify}\par

{\fontsize{10pt}{12.0pt}\selectfont O(1) $ \subset $  O(log\textit{n}) $ \subset $  O(\textit{n}) $ \subset $  O(\textit{n}log\textit{n}) $ \subset $  O(\textit{n}\textsuperscript{2}) $ \subset $  O(\textit{n}\textsuperscript{3}) $ \subset $  O(\textit{n}\textsuperscript{k}) $ \subset $  O(2\textit{\textsuperscript{n}}) $ \subset $  O(\textit{n}!). \par}\par

{\fontsize{10pt}{12.0pt}\selectfont \textbf{Solución al Problema 1.7 }(☺) \par}\par

{\fontsize{10pt}{12.0pt}\selectfont a) \textsubscript{que }Por \textit{\textsubscript{f}}\textsubscript{(\textit{n}) }la \textsubscript{$ \leq $  }definición \textit{\textsubscript{c}}\textsubscript{1\textit{g}(\textit{n}) para }de \textsubscript{todo }O, sabemos \textit{\textsubscript{n }}\textsubscript{$ \geq $  \textit{n}1. }que \textit{f}$ \in $ O(\textit{g}) si y sólo si existen \textit{c}\textsubscript{1 > 0 y \textit{n}1 tales > 0 }Análogamente, \textsubscript{y \textit{n}2 tales que \textit{g}(\textit{n}) }por \textsubscript{$ \geq $  }la \textit{\textsubscript{c}}definición \textsubscript{2\textit{f}(\textit{n}) para todo }de $ \Omega $  \textit{\textsubscript{n }}tenemos \textsubscript{$ \geq $  \textit{n}2. Por }que \textsubscript{consiguiente, }\textit{g}$ \in $ $ \Omega $ (\textit{f}) si y sólo si existen \textit{c}\textsubscript{2 }$ \Rightarrow $ ) Si \textit{f}$ \in $ O(\textit{g}) basta tomar \textit{c}\textsubscript{2=1/\textit{c}1 y \textit{n}2=\textit{n}1 para ver que \textit{g}(\textit{n})$ \in $ $ \Omega $ (\textit{f}). }$ \Leftarrow $ ) Recíprocamente, si \textit{g}$ \in $ $ \Omega $ (\textit{f}) basta tomar \textit{c}\textsubscript{1=1/\textit{c}2 y \textit{n}1=\textit{n}2 para que \textit{f}$ \in $ O(\textit{g}). cero, }Obsérvese \textsubscript{y por tanto }que \textsubscript{poseen }esto es \textsubscript{inverso. }posible pues \textit{c}\textsubscript{1 y \textit{c}2 son ambos estrictamente mayores que }b) Sean \textit{f}(\textit{n})=\textsuperscript{⎧ }\par}\par

{\fontsize{11pt}{13.2pt}\selectfont ⎨\textsubscript{⎩}si1 \par}\par

\textit{\textsuperscript{n }}\textsuperscript{2 }\par

\begin{Center}
{\fontsize{11pt}{13.2pt}\selectfont si \textit{\textsuperscript{n }}es par. \textit{n}es impar. {\fontsize{10pt}{12.0pt}\selectfont y \textit{g}(\textit{n})=\textit{n}\textsuperscript{2}. Entonces $ \Theta $ (\textit{g})=$ \Theta $ (\textit{n}\textsuperscript{2}), \textsubscript{Sin embargo, si \textit{n }es impar }y por otro lado O(\textit{f})=O(\textit{n}\textsuperscript{2}), \textsubscript{no puede existir \textit{c}>0 tal que }con \textit{\textsubscript{f}}\textsubscript{(\textit{n}) }lo \textsubscript{= 1 }cual \textsubscript{$ \geq $  \textit{cn}}\textit{f }{\fontsize{7pt}{8.4pt}\selectfont 2 {\fontsize{10pt}{12.0pt}\selectfont $ \in $ O(\textit{n}\textsubscript{= \textit{cg}(\textit{n}), }\textsuperscript{2})=O(\textit{g}). \textsubscript{y por consiguiente \textit{f}$ \notin $ $ \Omega $ (\textit{g}). }\par}\par}\par}\par}
\end{Center}\par

{\fontsize{10pt}{12.0pt}\selectfont 32 {\fontsize{7pt}{8.4pt}\selectfont TÉCNICAS DE DISEÑO DE ALGORITMOS \par}\par}\par

\begin{justify}
{\fontsize{10pt}{12.0pt}\selectfont Intuitivamente, lo que buscamos es una función \textit{f }cuyo crecimiento asintótico \textsubscript{estuviera que \textit{g}) y que acotado sin embargo superiormente \textit{f }no estuviera por \textit{g }acotado (es decir, inferiormente que \textit{f }no creciera por \textit{g}. $``$más deprisa$"$  }c) Veamos que log\textit{\textsubscript{an}}\textsubscript{$ \in $ $ \Theta $ (log\textit{bn}). }Sabemos por las propiedades de los logaritmos que si \textit{a }y \textit{b }son números reales \textsubscript{mayores que 1 se cumple que }\par}
\end{justify}\par

\textsubscript{log }{\fontsize{6pt}{7.2pt}\selectfont \textit{b n }\par}\par

\textsubscript{= }log \textit{\textsuperscript{a }}\par

\textit{\textsuperscript{n }}\textsubscript{log }\par

{\fontsize{6pt}{7.2pt}\selectfont \textit{ab}\textsubscript{. }\par}\par

{\fontsize{10pt}{12.0pt}\selectfont Con esto, \par}\par

\textit{\textsubscript{lim }}{\fontsize{6pt}{7.2pt}\selectfont \textit{n }\textsubscript{$\infty$ $ \rightarrow $  }log \textit{a }log \par}\par

\textit{\textsubscript{b}n \textsubscript{n}}\textsubscript{= \textit{n }}\textit{lim }\textsubscript{$\infty$ $ \rightarrow $  }\textsuperscript{log }\textit{\textsubscript{a }b }\textsubscript{= }\textsuperscript{log }\textit{\textsubscript{a }b }{\fontsize{10pt}{12.0pt}\selectfont , \par}\par

{\fontsize{10pt}{12.0pt}\selectfont que \textsubscript{$ \Theta $ (log}es \textit{\textsubscript{an}}\textsubscript{) }una \textsubscript{= $ \Theta $ (log}constante \textit{\textsubscript{bn}}\textsubscript{). }real finita distinta de cero (pues \textit{a,b}>1), y por tanto \textbf{Solución al }\par}\par

\begin{adjustwidth}{0.58in}{0.57in}
{\fontsize{10pt}{12.0pt}\selectfont De esta forma hemos ido desarrollando los términos de esta sucesión, cada uno \textsubscript{en función de términos anteriores. Sólo nos queda por calcular el número de términos (\textit{x}) que hemos tenido que desarrollar. }Pero ese número \textit{x }coincide con el número de términos de la sucesión \textit{n}/2, \textit{n}/4, \textit{\textsubscript{n}}\textsubscript{/8,...,4,2,1, que es log\textit{n }pues \textit{n }es una potencia de 2. En consecuencia, }\par}\par

\end{adjustwidth}

\begin{adjustwidth}{0.63in}{0.61in}
\textit{T}(\textit{n}) = 4\textsuperscript{log\textit{n }}\textit{T}(1)+ log\textit{n}⋅\textit{n}\textsuperscript{2 }= \textit{n }\textsuperscript{log 4 }⋅1+ log \textit{n}⋅\textit{n }\textsuperscript{2 }= \textit{n}\textsuperscript{2 }+log \textit{n}⋅\textit{n}\textsuperscript{2}{\fontsize{10pt}{12.0pt}\selectfont $ \in $ $ \Theta $ (\textit{n}\textsuperscript{2}log\textit{n}). \par}\par

\end{adjustwidth}

\begin{adjustwidth}{0.58in}{2.66in}
{\fontsize{10pt}{12.0pt}\selectfont d) \textit{T}(\textit{n}) = 2\textit{T}(\textit{n}/2) + \textit{n}log\textit{n }si \textit{n}>1, \textit{n }potencia de 2. \par}\par

\end{adjustwidth}

\begin{adjustwidth}{0.57in}{1.61in}
{\fontsize{10pt}{12.0pt}\selectfont Haciendo el cambio \textit{n }= 2\textit{\textsuperscript{k }}(o, lo que es igual, \textit{k }= log\textit{n}) obtenemos \par}\par

\end{adjustwidth}

\begin{adjustwidth}{0.58in}{2.46in}
{\fontsize{10pt}{12.0pt}\selectfont \textit{T}(2\textit{\textsuperscript{k}}) = 2\textit{T}(2\textit{\textsuperscript{k}}\textsuperscript{–1}) + \textit{k}2\textit{\textsuperscript{k}}. Llamando \textit{t\textsubscript{k }}\textsubscript{= \textit{T}(2}{\fontsize{7pt}{8.4pt}\selectfont \textit{k}\textsubscript{), la ecuación final es }\par}\par}\par

\end{adjustwidth}

\begin{adjustwidth}{0.58in}{0.57in}
\begin{Center}
{\fontsize{10pt}{12.0pt}\selectfont \textit{t\textsubscript{k }}\textsubscript{= 2\textit{tk}–1 + \textit{k}2}{\fontsize{7pt}{8.4pt}\selectfont \textit{k}\textsubscript{, }{\fontsize{10pt}{12.0pt}\selectfont ecuación en recurrencia no homogénea con ecuación característica (\textit{x}–2)\textsuperscript{3 }= 0. Por \textsubscript{tanto, }\par}\par}\par}
\end{Center}\par

\end{adjustwidth}

\begin{adjustwidth}{0.58in}{0.88in}
\begin{Center}
{\fontsize{10pt}{12.0pt}\selectfont \textit{t\textsubscript{k }}\textsubscript{= \textit{c}12}{\fontsize{7pt}{8.4pt}\selectfont \textit{k }\textsubscript{+ \textit{c}2\textit{k}2}\textit{k }\textsubscript{+ \textit{c}3\textit{k}}2\textsubscript{2}\textit{k}\textsubscript{. }{\fontsize{10pt}{12.0pt}\selectfont Necesitamos ahora deshacer los cambios hechos. Primero \textit{t\textsubscript{k }}\textsubscript{= \textit{T}(2}{\fontsize{7pt}{8.4pt}\selectfont \textit{k}\textsubscript{), con lo que }\par}\par}\par}\par}
\end{Center}\par

\end{adjustwidth}

\begin{adjustwidth}{0.58in}{2.27in}
{\fontsize{10pt}{12.0pt}\selectfont \textit{T}(2\textit{\textsuperscript{k}}) = \textit{c}\textsubscript{12}{\fontsize{7pt}{8.4pt}\selectfont \textit{k }\textsubscript{+ \textit{c}2\textit{k}2}\textit{k }\textsubscript{+ \textit{c}3\textit{k}}2\textsubscript{2}\textit{k}\textsubscript{, }{\fontsize{10pt}{12.0pt}\selectfont y después \textit{n }= 2\textit{\textsuperscript{k }}(\textit{k }= log\textit{n}), por lo cual \par}\par}\par}\par

\end{adjustwidth}

\begin{adjustwidth}{0.58in}{0.57in}
\begin{Center}
{\fontsize{10pt}{12.0pt}\selectfont \textit{T}(\textit{n}) = \textit{c}\textsubscript{1\textit{n }+ \textit{c}2\textit{n}log\textit{n }+ \textit{c}3\textit{n}log}{\fontsize{7pt}{8.4pt}\selectfont 2\textit{\textsubscript{n}}\textsubscript{. }{\fontsize{10pt}{12.0pt}\selectfont De esta ecuación no conocemos condiciones iniciales para calcular todas las \textsubscript{constantes, pero sí es posible intentar fijar alguna de ellas. Para eso, basta sustituir la expresión que hemos encontrado para \textit{T}(\textit{n}) en la ecuación original: }\par}\par}\par}
\end{Center}\par

\end{adjustwidth}

\begin{adjustwidth}{0.58in}{1.7in}
{\fontsize{10pt}{12.0pt}\selectfont \textit{n}log\textit{n }= \textit{T}(\textit{n}) – 2\textit{T}(\textit{n}/2) = (\textit{c}\textsubscript{3 – \textit{c}2)\textit{n }+ 2\textit{c}3\textit{n}log\textit{n}, }por lo que \textit{c}\textsubscript{3\textit{=c}2 y 2\textit{c}3=1, de donde }\par}\par

\end{adjustwidth}

\begin{adjustwidth}{0.58in}{0.63in}
\begin{Center}
{\fontsize{10pt}{12.0pt}\selectfont \textit{T}(\textit{n}) = \textit{c}\textsubscript{1\textit{n }+ 1/2\textit{n}log\textit{n }+ 1/2\textit{n}log}{\fontsize{7pt}{8.4pt}\selectfont 2\textit{\textsubscript{n}}\textsubscript{. }{\fontsize{10pt}{12.0pt}\selectfont En consecuencia \textit{T}(\textit{n})$ \in $ $ \Theta $ (\textit{n}log\textsuperscript{2}\textit{n}) independientemente de las condiciones iniciales. \par}\par}\par}
\end{Center}\par

\end{adjustwidth}

\begin{adjustwidth}{0.58in}{2.63in}
{\fontsize{10pt}{12.0pt}\selectfont e) \textit{T}(\textit{n}) = 3\textit{T}(\textit{n}/2) + 5\textit{n }+ 3 si \textit{n}>1, \textit{n }potencia de 2. \par}\par

\end{adjustwidth}

\begin{adjustwidth}{0.58in}{1.61in}
{\fontsize{10pt}{12.0pt}\selectfont Haciendo el cambio \textit{n }= 2\textit{\textsuperscript{k }}(o, lo que es igual, \textit{k }= log\textit{n}) obtenemos \par}\par

\end{adjustwidth}

\begin{adjustwidth}{0.58in}{2.32in}
{\fontsize{10pt}{12.0pt}\selectfont \textit{T}(2\textit{\textsuperscript{k}}) = 3\textit{T}(2\textit{\textsuperscript{k}}\textsuperscript{–1}) + 5$ \cdot $ 2\textit{\textsuperscript{k }}+ 3. Llamando \textit{t\textsubscript{k }}\textsubscript{= \textit{T}(2}{\fontsize{7pt}{8.4pt}\selectfont \textit{k}\textsubscript{), la ecuación final es: }\par}\par}\par

\end{adjustwidth}

{\fontsize{7pt}{8.4pt}\selectfont LA COMPLEJIDAD DE LOS ALGORITMOS {\fontsize{10pt}{12.0pt}\selectfont 37 \par}\par}\par

\begin{Center}
{\fontsize{10pt}{12.0pt}\selectfont \textit{t\textsubscript{k }}\textsubscript{= 3\textit{tk}–1 + 5$ \cdot $ 2}{\fontsize{7pt}{8.4pt}\selectfont \textit{k }\textsubscript{+ 3, }{\fontsize{10pt}{12.0pt}\selectfont ecuación en recurrencia no homogénea cuya ecuación característica asociada es \textsubscript{(\textit{x}–3)(\textit{x}–2)(\textit{x}–1) = 0. Por tanto, }\par}\par}\par}
\end{Center}\par

\begin{Center}
{\fontsize{10pt}{12.0pt}\selectfont \textit{t\textsubscript{k }}\textsubscript{= \textit{c}13}{\fontsize{7pt}{8.4pt}\selectfont \textit{k }\textsubscript{+ \textit{c}22}\textit{k }\textsubscript{+ \textit{c}3. }{\fontsize{10pt}{12.0pt}\selectfont Necesitamos ahora deshacer los cambios hechos. Primero \textit{t\textsubscript{k }}\textsubscript{= \textit{T}(2}{\fontsize{7pt}{8.4pt}\selectfont \textit{k}\textsubscript{), con lo que }\par}\par}\par}\par}
\end{Center}\par

{\fontsize{10pt}{12.0pt}\selectfont \textit{T}(2\textit{\textsuperscript{k}}) = \textit{c}\textsubscript{13}{\fontsize{7pt}{8.4pt}\selectfont \textit{k }\textsubscript{+ \textit{c}22}\textit{k }\textsubscript{+ \textit{c}3 }{\fontsize{10pt}{12.0pt}\selectfont y después \textit{n }= 2\textit{\textsuperscript{k }}(\textit{k }= log\textit{n}), por lo cual \par}\par}\par}\par

\begin{Center}
{\fontsize{10pt}{12.0pt}\selectfont \textit{T}(\textit{n}) = \textit{c}\textsubscript{13}{\fontsize{7pt}{8.4pt}\selectfont log\textit{n }\textsubscript{+ \textit{c}2\textit{n }+ \textit{c}3 = \textit{c}1\textit{n}}log3 \textsubscript{+ \textit{c}2\textit{n }+ \textit{c}3. constantes, }{\fontsize{10pt}{12.0pt}\selectfont De esta \textsubscript{pero }ecuación \textsubscript{sí es }no \textsubscript{posible }conocemos \textsubscript{intentar }condiciones \textsubscript{fijar alguna }iniciales \textsubscript{de ellas. }para \textsubscript{Para }calcular \textsubscript{eso basta }todas \textsubscript{sustituir }las \textsubscript{la expresión que hemos encontrado para \textit{T}(\textit{n}) en la ecuación en recurrencia original, y obtenemos: }\par}\par}\par}
\end{Center}\par

{\fontsize{10pt}{12.0pt}\selectfont c\textsubscript{1\textit{n}}{\fontsize{7pt}{8.4pt}\selectfont log3 \textsubscript{+ \textit{c}2\textit{n }+ \textit{c}3 = 3(\textit{c}1(\textit{n}}log3\textsubscript{/3) + \textit{c}2\textit{n}/2 + \textit{c}3) + 5\textit{n }+ 3. }{\fontsize{10pt}{12.0pt}\selectfont Igualando los coeficientes de \textit{n}\textsuperscript{log3}, \textit{n }y los términos independientes obtenemos \textit{\textsubscript{c}}\textsubscript{3=–3/2 y \textit{c}2=–10, de donde }\textit{T}(\textit{n}) = \textit{c}\textsubscript{1\textit{n}}{\fontsize{7pt}{8.4pt}\selectfont log3 \textsubscript{– 10\textit{n }– 3/2. \textit{T}(\textit{n}) }{\fontsize{10pt}{12.0pt}\selectfont Como \textsubscript{$ \in $ $ \Theta $ (\textit{n}) }log3 \textsubscript{si \textit{c}}> \textsubscript{1 = }1, \textsubscript{0. }\textit{T}(\textit{n}) será de complejidad $ \Theta $ (\textit{n}\textsuperscript{log3}) si \textit{c}\textsubscript{1 es distinto de cero, o bien }Para \textsubscript{iniciales }ver \textsubscript{que le }cuándo \textsubscript{hacen }\textit{c}\textsubscript{tomar 1 vale cero estudiaremos los valores de las condiciones ese valor, en este caso \textit{T}(1). Por un lado, utilizando la ecuación original, tenemos que para \textit{n }= 2: }\par}\par}\par}\par}\par}\par

{\fontsize{10pt}{12.0pt}\selectfont \textit{T}(2) = 3\textit{T}(1) + 10 + 3. \par}\par

{\fontsize{10pt}{12.0pt}\selectfont Por otro lado, basándonos en la ecuación que hemos obtenido, \par}\par


\vspace{\baselineskip}
{\fontsize{10pt}{12.0pt}\selectfont Haciendo el cambio \textit{n }= 2\textit{\textsuperscript{k }}(o, lo que es igual, \textit{k }= log\textit{n}) obtenemos \par}\par

{\fontsize{10pt}{12.0pt}\selectfont \textit{T}(2\textit{\textsuperscript{k}}) = 2\textit{T}(2\textit{\textsuperscript{k}}\textsuperscript{–1}) + \textit{k}. Llamando \textit{t\textsubscript{k }}\textsubscript{= \textit{T}(2}{\fontsize{7pt}{8.4pt}\selectfont \textit{k}\textsubscript{), la ecuación final es }\par}\par}\par

{\fontsize{10pt}{12.0pt}\selectfont 38 {\fontsize{7pt}{8.4pt}\selectfont TÉCNICAS DE DISEÑO DE ALGORITMOS \par}\par}\par

\begin{Center}
{\fontsize{10pt}{12.0pt}\selectfont \textit{t\textsubscript{k }}\textsubscript{= 2\textit{tk}–1 + \textit{k}, }ecuación en recurrencia no homogénea que puede ser expresada como \par}
\end{Center}\par

\begin{Center}
{\fontsize{10pt}{12.0pt}\selectfont \textit{t\textsubscript{k }}\textsubscript{– 2\textit{tk}–1 = \textit{k }}y cuya ecuación característica asociada es (\textit{x}–2)(\textit{x}–1)\textsuperscript{2 }= 0. Por tanto, \par}
\end{Center}\par

\begin{Center}
{\fontsize{10pt}{12.0pt}\selectfont \textit{t\textsubscript{k }}\textsubscript{= \textit{c}12}{\fontsize{7pt}{8.4pt}\selectfont \textit{k }\textsubscript{+ \textit{c}2 + \textit{c}3\textit{k}. }{\fontsize{10pt}{12.0pt}\selectfont Necesitamos ahora deshacer los cambios hechos. Primero \textit{t\textsubscript{k }}\textsubscript{= \textit{T}(2}{\fontsize{7pt}{8.4pt}\selectfont \textit{k}\textsubscript{), con lo que }\par}\par}\par}\par}
\end{Center}\par

{\fontsize{10pt}{12.0pt}\selectfont \textit{T}(2\textit{\textsuperscript{k}}) = \textit{c}\textsubscript{12}{\fontsize{7pt}{8.4pt}\selectfont \textit{k }\textsubscript{+ \textit{c}2 + \textit{c}3\textit{k }}{\fontsize{10pt}{12.0pt}\selectfont y después \textit{n }= 2\textit{\textsuperscript{k }}(\textit{k }= log\textit{n}), y por tanto \par}\par}\par}\par

{\fontsize{10pt}{12.0pt}\selectfont \textit{T}(\textit{n}) = \textit{c}\textsubscript{1\textit{n }+ \textit{c}2 + \textit{c}3log\textit{n}. }De esta ecuación no conocemos condiciones iniciales para calcular todas las \textsubscript{constantes, pero sí es posible intentar fijar alguna de ellas. Para eso, basta sustituir la original, expresión y obtenemos: que hemos encontrado para \textit{T}(\textit{n}) en la ecuación en recurrencia }\textit{c}\textsubscript{1\textit{n }+ \textit{c}2 + \textit{c}3log\textit{n }= 2(\textit{c}1\textit{n}/2 + \textit{c}2 + \textit{c}3log\textit{n }– \textit{c}3) + log\textit{n}. \textit{c}3=–1 }Igualando \textsubscript{y \textit{c}2=–2, }los \textsubscript{de }coeficientes \textsubscript{donde }\par}\par

{\fontsize{10pt}{12.0pt}\selectfont de log\textit{n }y los términos independientes obtenemos que \textit{T}(\textit{n}) = \textit{c}\textsubscript{1\textit{n }– 2 – log\textit{n}. \textit{T}(\textit{n}) }Esta \textsubscript{$ \in $ $ \Theta $ (log\textit{n}) }función \textsubscript{si }será \textit{\textsubscript{c}}\textsubscript{1 = }de \textsubscript{0. }orden de complejidad $ \Theta $ (\textit{n}) si \textit{c}\textsubscript{1 es distinto de cero, o bien iniciales }Para ver \textsubscript{que le }cuándo \textsubscript{hacen }\textit{c}\textsubscript{tomar 1 vale ese cero valor, estudiaremos en este caso los \textit{T}(1). valores Por un de lado, las utilizando condiciones la ecuación original, tenemos que para \textit{n }= 2: }\par}\par

{\fontsize{10pt}{12.0pt}\selectfont \textit{T}(2) = 2\textit{T}(1) + 1. \par}\par

{\fontsize{10pt}{12.0pt}\selectfont Por otro lado, basándonos en la ecuación que hemos obtenido \par}\par

\begin{Center}
{\fontsize{10pt}{12.0pt}\selectfont \textit{T}(2) = 2\textit{c}\textsubscript{1 – 2 – 1. }Igualando ambas ecuaciones, obtenemos que \textit{c}\textsubscript{1 = T(1) + 2. Por tanto, }\par}
\end{Center}\par

\begin{FlushRight}
\textsuperscript{⎧ ⎨}⎩$ \Theta $  $-$ $ \neq $  \textit{\textsubscript{nT }}
\end{FlushRight}\par

\begin{Center}
\textsubscript{)( $ \in $  }$ \Theta $  si)(log \textsubscript{)( \textit{n }}\textit{Tn }2)1( $-$ = \textsubscript{si \textit{T }2)1( }{\fontsize{10pt}{12.0pt}\selectfont g) \textit{T}(\textit{n}) = 2\textit{T}(\textit{n}\textsuperscript{1/2}) + log\textit{n }con \textit{n}=2\textsuperscript{2\textit{k}}; \textit{T}(2)=1. Haciendo el cambio \textit{n }= 2\textsuperscript{2\textit{k }}(\textit{k }= loglog\textit{n}) obtenemos la ecuación \par}
\end{Center}\par

{\fontsize{10pt}{12.0pt}\selectfont \textit{T}(2\textsuperscript{2\textit{k}}) = 2\textit{T}(2\textsuperscript{2\textit{k}–1}) + log2\textsuperscript{2\textit{k}}. \par}\par

{\fontsize{7pt}{8.4pt}\selectfont LA COMPLEJIDAD DE LOS ALGORITMOS {\fontsize{10pt}{12.0pt}\selectfont 39 \par}\par}\par

{\fontsize{10pt}{12.0pt}\selectfont Llamando \textit{t\textsubscript{k }}\textsubscript{= \textit{T}(2}2\textit{\textsuperscript{k}}), la ecuación final es \par}\par

\begin{Center}
{\fontsize{10pt}{12.0pt}\selectfont \textit{t\textsubscript{k }}\textsubscript{= 2\textit{tk}–1 + 2}{\fontsize{7pt}{8.4pt}\selectfont \textit{k}\textsubscript{, }{\fontsize{10pt}{12.0pt}\selectfont ecuación en recurrencia no homogénea cuya ecuación característica es (\textit{x}–2)\textsuperscript{2 }= 0. \textsubscript{Por tanto, }\par}\par}\par}
\end{Center}\par

\begin{Center}
{\fontsize{10pt}{12.0pt}\selectfont \textit{t\textsubscript{k }}\textsubscript{= \textit{c}12}{\fontsize{7pt}{8.4pt}\selectfont \textit{k }\textsubscript{+ \textit{c}2\textit{k}2}\textit{k}\textsubscript{. }{\fontsize{10pt}{12.0pt}\selectfont Necesitamos ahora deshacer los cambios hechos. Primero \textit{t\textsubscript{k }}\textsubscript{= \textit{T}(2}2\textit{\textsuperscript{k}}), con lo que \par}\par}\par}
\end{Center}\par

\begin{Center}
{\fontsize{10pt}{12.0pt}\selectfont \textit{T}(2\textsuperscript{2\textit{k}}) = \textit{c}\textsubscript{12}{\fontsize{7pt}{8.4pt}\selectfont \textit{k }\textsubscript{+ \textit{c}2\textit{k}2}\textit{k }{\fontsize{10pt}{12.0pt}\selectfont y después \textit{n }= 2\textsuperscript{2\textit{k }}(\textit{k }= loglog\textit{n}, o bien log\textit{n }= 2\textit{\textsuperscript{k}}), por lo cual tenemos que \par}\par}\par}
\end{Center}\par

\begin{Center}
{\fontsize{10pt}{12.0pt}\selectfont \textit{T}(\textit{n}) = \textit{c}\textsubscript{1log\textit{n }+ \textit{c}2log\textit{n}$ \cdot $ loglog\textit{n}. }Para calcular las constantes necesitamos las condiciones iniciales. Como \textsubscript{disponemos de sólo una y tenemos dos incógnitas, usamos la ecuación original para obtener la otra: }\par}
\end{Center}\par

{\fontsize{10pt}{12.0pt}\selectfont \textit{T}(4) = 2\textit{T}(2) + log4 = 4. \par}\par

{\fontsize{10pt}{12.0pt}\selectfont \textbf{Solución al Problema 1.10 }( ) \par}\par

{\fontsize{10pt}{12.0pt}\selectfont 44 {\fontsize{7pt}{8.4pt}\selectfont TÉCNICAS DE DISEÑO DE ALGORITMOS \par}\par}\par

\begin{justify}
{\fontsize{10pt}{12.0pt}\selectfont a) Cierto. Se deduce de la propiedad 6 del apartado 1.3.1, pero veamos una posible \textsubscript{demostración directa: \textit{n }para $ \geq $  }Si \textit{\textsubscript{nn }}\textsubscript{1. }\textit{T}\textsubscript{$ \geq $  1$ \in $ O(\textit{f}), Análogamente, \textit{n}2. sabemos que como existen \textit{T}2$ \in $ O(\textit{f}), \textit{c}1 existen > 0 y \textit{n}1 \textit{c}2 tales > 0 y que \textit{n}2 \textit{T}tales 1(\textit{n}) $ \leq $  que \textit{c}1\textit{f}(\textit{n}) \textit{T}2(\textit{n}) para $ \leq $  \textit{c}[1.1] todo 2\textit{f}(\textit{n}) un número }Para comprobar \textsubscript{natural \textit{n}}que \textsubscript{0 tales }\textit{T}\textsubscript{1 + que \textit{T}2 \textit{T}$ \in $ O(\textit{f}), 1(\textit{n}) + debemos \textit{T}2(\textit{n}) $ \leq $  \textit{cf}(\textit{n}) encontrar para todo una \textit{n }constante $ \geq $  \textit{n}0. real \textit{c }> [1.2] 0 y verifica }Apoyándonos \textsubscript{la ecuación }en \textsubscript{[1.2] }[1.1], \textsubscript{para }basta \textsubscript{todo }tomar \textit{\textsubscript{n }}\textsubscript{$ \geq $  }\textit{n\textsubscript{n}}\textsubscript{0 0. = \textit{máx}$ \{ $ \textit{n}1,\textit{n}2$ \} $  y \textit{c }= \textit{c}1 + \textit{c}2, con las que se existan, }Existe \textsubscript{como }otra \textsubscript{sucede }forma \textsubscript{por }de \textsubscript{ejemplo }demostrarlo, \textsubscript{cuando }utilizando \textsubscript{las funciones }límites \textsubscript{son continuas: }en caso de que estos Si \textit{T}\textsubscript{1$ \in $ O(\textit{f}), entonces \textit{lim n }$\infty$ $ \rightarrow $  })(\textsubscript{)( }\par}
\end{justify}\par

\textit{nT }\textsubscript{1 }\textit{nf }\par

{\fontsize{10pt}{12.0pt}\selectfont = \textit{k}\textsubscript{1<$\infty$ . }\par}\par

{\fontsize{10pt}{12.0pt}\selectfont Análogamente, como \textit{T}\textsubscript{2$ \in $  O(\textit{f}), \textit{lim n }$\infty$ $ \rightarrow $  }\textit{nT nf }\par}\par

\textsubscript{2 }\par

\textsubscript{)( })( {\fontsize{10pt}{12.0pt}\selectfont = \textit{k}\textsubscript{2 <$\infty$ . [1.3] }\par}\par

{\fontsize{10pt}{12.0pt}\selectfont Veamos entonces que \textit{\textsubscript{lim n }}\par}\par

{\fontsize{6pt}{7.2pt}\selectfont $\infty$ $ \rightarrow $  \textit{nTnT }\textsubscript{1 }\par}\par

\begin{Center}
)()( + \textsubscript{2 }\textit{nf }\textsubscript{)( }
\end{Center}\par

\textsuperscript{= \textit{k}<$\infty$ . [1.4] }\par

\textsubscript{podemos }{\fontsize{10pt}{12.0pt}\selectfont Pero [1.4] \textsubscript{conmutar }es cierto \textsubscript{la suma }pues, \textsubscript{con }como \textsubscript{el límite }los dos \textsubscript{y obtenemos }límites en \textsubscript{que }[1.3] son finitos y positivos \textit{\textsubscript{lim n }}\textsubscript{$\infty$ $ \rightarrow $  }\textit{nTnT }\textsubscript{1 }\par}\par

\begin{Center}
)()( \textsuperscript{+ }\textsubscript{2 \textit{nf }})( 
\end{Center}\par

= \textit{\textsubscript{lim n }}\textsubscript{$\infty$ $ \rightarrow $  }\textit{nT \textsubscript{nf }}\textsubscript{1 })( )( \par

+ \textit{\textsubscript{n lim }}\textsubscript{$\infty$ $ \rightarrow $  }\textit{nT }\textsubscript{2 })( \par

\textit{nf }\par

\textsubscript{)( }\par

{\fontsize{10pt}{12.0pt}\selectfont = \textit{k}\textsubscript{1 + \textit{k}2 <$\infty$ . }\par}\par

{\fontsize{10pt}{12.0pt}\selectfont b) Cierto. \par}\par

{\fontsize{10pt}{12.0pt}\selectfont Análogamente a lo realizado en el apartado anterior, si \textit{T}\textsubscript{1$ \in $ O(\textit{f}), entonces \textit{lim n }$\infty$ $ \rightarrow $  }\textit{nT nf }\par}\par

\textsubscript{1 }\par

)( )({\fontsize{10pt}{12.0pt}\selectfont =\textit{k}\textsubscript{1 < $\infty$ . Igualmente, como \textit{T}2$ \in $ O(\textit{f}), \textit{lim n }$\infty$ $ \rightarrow $  }\textit{nT nf }\par}\par

\textsubscript{2 }\par

\textsubscript{)( })( {\fontsize{10pt}{12.0pt}\selectfont = \textit{k}\textsubscript{2 < $\infty$ . [1.5] }\par}\par

{\fontsize{10pt}{12.0pt}\selectfont Veamos entonces que \textit{\textsubscript{lim n }}\par}\par

{\fontsize{6pt}{7.2pt}\selectfont $\infty$ $ \rightarrow $  \textit{nTnT }\textsubscript{1 }\par}\par

\begin{Center}
)()( $-$  \textsubscript{2 }\textit{nf }\textsubscript{)( }
\end{Center}\par

\textsuperscript{= \textit{k }< $\infty$ . [1.6] }\par

\textsubscript{positivos }{\fontsize{10pt}{12.0pt}\selectfont Pero [1.6] \textsubscript{podemos }es cierto \textsubscript{conmutar }pues, \textsubscript{la }como \textsubscript{resta }los \textsubscript{con }dos \textsubscript{el límite }límites \textsubscript{y }en \textsubscript{obtenemos }[1.5] existen \textsubscript{que }\par}\par

{\fontsize{10pt}{12.0pt}\selectfont y son finitos y \textit{\textsubscript{lim n }}\textsubscript{$\infty$ $ \rightarrow $  }\textit{nTnT }\textsubscript{1 }\par}\par

\begin{Center}
)()( \textsuperscript{$-$  }\textsubscript{2 \textit{nf }})( 
\end{Center}\par

\begin{FlushRight}
= \textit{\textsubscript{lim n }}\textsubscript{$\infty$ $ \rightarrow $  }\textit{nT \textsubscript{nf }}\textsubscript{1 })( )( 
\end{FlushRight}\par

\begin{FlushRight}
$-$  \textit{\textsubscript{lim n }}\textsubscript{$\infty$ $ \rightarrow $  }\textit{nT nf }\textsubscript{2 })( )( 
\end{FlushRight}\par

{\fontsize{10pt}{12.0pt}\selectfont = \textit{k}\textsubscript{1 \textit{– k}2 < $\infty$ . }\par}\par

{\fontsize{7pt}{8.4pt}\selectfont LA COMPLEJIDAD DE LOS ALGORITMOS {\fontsize{10pt}{12.0pt}\selectfont 45 \par}\par}\par

{\fontsize{10pt}{12.0pt}\selectfont c) Falso. \par}\par

\textsubscript{y \textit{T}}{\fontsize{10pt}{12.0pt}\selectfont Consideremos \textsubscript{2$ \in $ O(\textit{f}), pero sin }\textit{T}\textsubscript{1(\textit{n}) embargo = \textit{n}}{\fontsize{7pt}{8.4pt}\selectfont 2\textsubscript{, \textit{TT}1(\textit{n})/\textit{T}2(\textit{n}) = \textit{n}, 2(\textit{n}) y \textit{f}(\textit{n}) = = \textit{n}$ \notin $ O(1). \textit{n}}3\textsubscript{. Tenemos por tanto que \textit{T}1$ \in $ O(\textit{f}) }{\fontsize{10pt}{12.0pt}\selectfont d) Falso. \par}\par}\par}\par

{\fontsize{10pt}{12.0pt}\selectfont Consideremos \textit{\textsubscript{T}}\textsubscript{1$ \in $ O(\textit{f}) y \textit{T}2$ \in $ O(\textit{f}), }de \textsubscript{pero }nuevo \textsubscript{sin }\textit{T}\textsubscript{embargo 1(\textit{n}) = \textit{n}}{\fontsize{7pt}{8.4pt}\selectfont 2\textsubscript{, \textit{TT}1$ \notin $ O(\textit{T}2(\textit{n}) = 2) \textit{n}, pues y \textit{f}(\textit{n}) \textit{n}}2\textsubscript{= $ \notin $ $O$ (\textit{n}). \textit{n}}3\textsubscript{. Tenemos por tanto que }{\fontsize{10pt}{12.0pt}\selectfont \textbf{Solución al Problema 1.11 }(☺) \par}\par}\par}\par

{\fontsize{10pt}{12.0pt}\selectfont Sean \textit{f}(\textit{n}) = \textit{n }y \textit{g}(\textit{n}) = \textsuperscript{⎧ }\par}\par

{\fontsize{11pt}{13.2pt}\selectfont ⎨\textsubscript{⎩}si1, \par}\par

{\fontsize{11pt}{13.2pt}\selectfont es impar. \textit{\textsuperscript{n }}\par}\par

\begin{Center}
\textsuperscript{2 }{\fontsize{11pt}{13.2pt}\selectfont si,\textit{\textsuperscript{n }}es par. \textit{n}{\fontsize{10pt}{12.0pt}\selectfont Si \textit{n }es impar, no podemos encontrar ninguna constante \textit{c }tal que \par}\par}
\end{Center}\par

{\fontsize{10pt}{12.0pt}\selectfont \textit{f}(\textit{n}) = \textit{n }$ \leq $  \textit{cg}(\textit{n}) = \textit{c}, \par}\par

{\fontsize{10pt}{12.0pt}\selectfont y por tanto \textit{f}$ \notin $ O(\textit{g}). Por otro lado, si \textit{n }es par no podemos encontrar ninguna \textsubscript{constante \textit{c }tal que }\par}\par

{\fontsize{10pt}{12.0pt}\selectfont \textit{g}(\textit{n}) = \textit{n}\textsuperscript{2 }$ \leq $  \textit{cf}(\textit{n}) = \textit{cn}, \par}\par

{\fontsize{10pt}{12.0pt}\selectfont y por tanto \textit{g}$ \notin $ O(\textit{f}). \par}\par

{\fontsize{10pt}{12.0pt}\selectfont \textbf{Solución al Problema 1.12 }(☺) \par}\par

{\fontsize{10pt}{12.0pt}\selectfont Para comprobar que log\textit{\textsuperscript{k}n}$ \in $ O(\textit{n}) basta ver que \par}\par

\textit{\textsubscript{lim }}{\fontsize{7pt}{8.4pt}\selectfont \textit{n }\par}\par

{\fontsize{6pt}{7.2pt}\selectfont $\infty$ $ \rightarrow $  log \par}\par

\begin{Center}
{\fontsize{7pt}{8.4pt}\selectfont \textit{k n n }\par}
\end{Center}\par

\textsuperscript{= 0 }\par

{\fontsize{10pt}{12.0pt}\selectfont para \textsubscript{log}{\fontsize{7pt}{8.4pt}\selectfont \textit{k\textsubscript{n}}\textsubscript{$ \notin $ $ \Omega $ (\textit{n}) }{\fontsize{10pt}{12.0pt}\selectfont todo \textit{k}. Pero \textsubscript{para cualquier }eso es cierto \textit{\textsubscript{k }}\textsubscript{> }siempre. \textsubscript{0. }Obsérvese además que por esa misma razón \textbf{Solución al Problema 1.13 }\par}\par}\par}\par

{\fontsize{10pt}{12.0pt}\selectfont Vamos a suponer que los tiempos de ejecución de las funciones \textit{Esvacio}, \textit{Izq}, \textit{Der }y \textit{\textsubscript{Raiz }}\textsubscript{es de \textit{c }operaciones elementales (OE), que el tiempo de ejecución de \textit{Opera }es \textit{d }OE, y el de \textit{Max2 }es 1 OE. }\par}\par

{\fontsize{10pt}{12.0pt}\selectfont \textbf{Procedimiento \textit{Inorden }}(☺) \par}\par

\begin{adjustwidth}{0.58in}{0.57in}
{\fontsize{10pt}{12.0pt}\selectfont 46 {\fontsize{7pt}{8.4pt}\selectfont TÉCNICAS DE DISEÑO DE ALGORITMOS \par}\par}\par

\end{adjustwidth}

\begin{adjustwidth}{0.58in}{0.57in}
\begin{justify}
{\fontsize{10pt}{12.0pt}\selectfont Para calcular el tiempo de ejecución, calcularemos primero el número de \textsubscript{operaciones elementales (OE) que se realizan: }– En la línea (1) se ejecutan 2+\textit{c }OE: la llamada a \textit{Esvacio }(1 OE), el tiempo de \textsubscript{ejecución de este procedimiento (\textit{c}) y una negación. }– En la línea (2) se efectúa la llamada a \textit{Izq }(1 OE), lo que tarda ésta en ejecutarse \textsubscript{(\textit{c }OE) más la llamada a \textit{Inorden }(1 OE) y lo que tarde ésta en ejecutarse, que va a depender del número de elementos del árbol \textit{Izq(t)}. }– En la línea (3) se ejecutan 2+\textit{c}+\textit{d }OE: dos llamadas a procedimientos y sus \textsubscript{respectivos tiempos de ejecución. }– El número de OE de la línea (4) se calcula de forma análoga a la línea (2): 2+\textit{c }\textsubscript{más lo que tarda \textit{Inorden }en ejecutarse con el número de elementos que hay en \textit{Der(t)}. }Para estudiar el tiempo de ejecución, vamos a considerar dos casos extremos: \textsubscript{que el árbol sea degenerado (es decir, una lista) y que sea equilibrado. Cualquier árbol se encuentra en una situación intermedia a estos dos casos. }\par}
\end{justify}\par

\end{adjustwidth}

\begin{adjustwidth}{0.58in}{0.57in}
{\fontsize{10pt}{12.0pt}\selectfont $\bullet$  Si \textit{t }es degenerado, podemos suponer sin pérdida de generalidad que \textit{\textsubscript{Esvacio(Izq(t)) }}\textsubscript{y que para todo \textit{a }subárbol de \textit{t }se verifica que \textit{Esvacio(Izq(a))}. Por tanto, el número de OE que se realizan en la ejecución de \textit{Inorden(t) }para un árbol \textit{t }con \textit{n }elementos es: }\par}\par

\end{adjustwidth}

\begin{adjustwidth}{1.17in}{1.72in}
{\fontsize{10pt}{12.0pt}\selectfont \textit{T}(\textit{n}) = (2+\textit{c}) + (2+\textit{c}+\textit{T}(0)) + (2+\textit{c}+\textit{d}) + (2+\textit{c}+\textit{T}(\textit{n}–1)) = \textsubscript{8+4\textit{c}+\textit{d}+\textit{T}(0)+\textit{T}(\textit{n}–1). \textit{T}(0) = 2+\textit{c}. }\par}\par

\end{adjustwidth}

\begin{adjustwidth}{0.77in}{0.57in}
{\fontsize{10pt}{12.0pt}\selectfont Con esto, \textit{T}(\textit{n}) = 10 + 5\textit{c }+ \textit{d }+ \textit{T}(\textit{n}–1), ecuación en recurrencia no homogénea \textsubscript{que podemos resolver desarrollándola telescópicamente: }\par}\par

\end{adjustwidth}

\begin{adjustwidth}{0.87in}{0.78in}
{\fontsize{10pt}{12.0pt}\selectfont \textit{T}(\textit{n}) = 10 + 5\textit{c }+ \textit{d }+ \textit{T}(\textit{n}–1) = (10 + 5\textit{c }+ \textit{d}) + (10 + 5\textit{c }+ \textit{d}) + \textit{T}(\textit{n}–2) = ... = \textsubscript{(10 + 5\textit{c }+ \textit{d})\textit{n }+ (2+\textit{c}) $ \in $ $ \Theta $ (\textit{n}) }\par}\par

\end{adjustwidth}

\begin{adjustwidth}{0.57in}{0.57in}
{\fontsize{10pt}{12.0pt}\selectfont $\bullet$  Si \textit{t }es equilibrado sus dos subárboles (izquierdo y derecho) tienen del orden de \textit{\textsubscript{n}}\textsubscript{/2 elementos y son a su vez equilibrados\textit{. }Por tanto, el número de OE que se realizan en la ejecución de \textit{Inorden(t) }para un árbol \textit{t }con \textit{n }elementos es: }\par}\par

\end{adjustwidth}

\begin{adjustwidth}{0.97in}{0.78in}
{\fontsize{10pt}{12.0pt}\selectfont \textit{T}(\textit{n})= (2+\textit{c}) + (2+\textit{c}+\textit{T}(\textit{n}/2)) + (2+\textit{c}+\textit{d}) + (2+\textit{c}+\textit{T}(\textit{n}/2)) = 8+4\textit{c}+\textit{d}+2\textit{T}(\textit{n}/2). \textit{\textsubscript{T}}\textsubscript{(0)= 2+\textit{c}. }\par}\par

\end{adjustwidth}

\begin{adjustwidth}{0.77in}{0.57in}
{\fontsize{10pt}{12.0pt}\selectfont Para resolver esta ecuación en recurrencia se hace el cambio \textit{t\textsubscript{k }}\textsubscript{= \textit{T}(2}{\fontsize{7pt}{8.4pt}\selectfont \textit{k}\textsubscript{), con lo que obtenemos }\par}\par}\par

\end{adjustwidth}

\begin{adjustwidth}{0.77in}{0.74in}
\begin{Center}
{\fontsize{10pt}{12.0pt}\selectfont \textit{t\textsubscript{k }}\textsubscript{– 2\textit{tk}–1 = 8 + 4\textit{c }+ \textit{d}, }ecuación no homogénea con ecuación característica (\textit{x}–2)(\textit{x}–1) = 0. Por tanto, \par}
\end{Center}\par

\end{adjustwidth}

\begin{adjustwidth}{0.77in}{2.76in}
{\fontsize{10pt}{12.0pt}\selectfont \textit{t\textsubscript{k }}\textsubscript{= \textit{c}12}{\fontsize{7pt}{8.4pt}\selectfont \textit{k }\textsubscript{+ \textit{c}2 }{\fontsize{10pt}{12.0pt}\selectfont y, deshaciendo los cambios, \par}\par}\par}\par

\end{adjustwidth}

\begin{adjustwidth}{2.67in}{2.66in}
{\fontsize{10pt}{12.0pt}\selectfont \textit{T}(\textit{n}) = \textit{c}\textsubscript{1\textit{n }+ \textit{c}2. }\par}\par

\end{adjustwidth}

\begin{adjustwidth}{0.58in}{0.57in}
{\fontsize{7pt}{8.4pt}\selectfont LA COMPLEJIDAD DE LOS ALGORITMOS {\fontsize{10pt}{12.0pt}\selectfont 47 \par}\par}\par

\end{adjustwidth}

\begin{adjustwidth}{0.77in}{0.57in}
{\fontsize{10pt}{12.0pt}\selectfont Para calcular las constantes, nos apoyamos en la condición inicial \textit{T}(0)=2+\textit{c}, \textsubscript{junto con el valor de \textit{T}(1), que puede ser calculado basándonos en la expresión de la ecuación en recurrencia: \textit{T}(1) = 8 + 4\textit{c }+ \textit{d }+ 2(2 + \textit{c}), obteniendo }\par}\par

\end{adjustwidth}

\begin{adjustwidth}{1.98in}{1.97in}
{\fontsize{10pt}{12.0pt}\selectfont \textit{T}(\textit{n}) = (10 + 5\textit{c }+ \textit{d})\textit{n }+ (2+\textit{c}) $ \in $  $ \Theta $ (\textit{n}). \par}\par

\end{adjustwidth}

\begin{adjustwidth}{0.57in}{0.57in}
{\fontsize{10pt}{12.0pt}\selectfont \textbf{Función \textit{Altura }}( ) Para determinar el tiempo de ejecución, calcularemos primero el número de \textsubscript{operaciones elementales (OE) que se realizan: }– En la línea (1) se ejecutan 1+\textit{c }OE: la llamada a \textit{Esvacio }(1 OE) más el tiempo \textsubscript{de ejecución de este procedimiento (\textit{c }OE). }– En la línea (2) se realiza 1 OE. – En la línea (4) se efectúan: \par}\par

\end{adjustwidth}

\begin{adjustwidth}{0.57in}{0.57in}
\begin{FlushRight}
{\fontsize{10pt}{12.0pt}\selectfont a) la llamada a \textit{Izq }(1 OE), lo que tarda ésta en ejecutarse (\textit{c }OE) más la llamada \textsubscript{a \textit{Altura }(1 OE) y lo que tarde ésta en ejecutarse, que va a depender del número de elementos del árbol \textit{Izq(t)}; más }b) la llamada a \textit{Der }(1 OE), lo que tarda ésta en ejecutarse (\textit{c }OE) más la \textsubscript{llamada a \textit{Altura }(1 OE) y lo que tarde ésta en ejecutarse, que va a depender del número de elementos del árbol \textit{Der(t)}; más }c) el cálculo del máximo de ambos números (1 OE), un incremento (1 OE) y el \textit{\textsubscript{RETURN }}\textsubscript{(1 OE). }Para estudiar el tiempo de ejecución de esta función consideraremos los mismos \textsubscript{casos que para la función \textit{Inorden}: que el árbol sea degenerado (es decir, una lista) o que sea equilibrado. }\par}
\end{FlushRight}\par

\end{adjustwidth}

\begin{adjustwidth}{0.57in}{0.57in}
{\fontsize{10pt}{12.0pt}\selectfont $\bullet$  Si \textit{t }es degenerado, podemos suponer sin pérdida de generalidad que \textit{\textsubscript{Esvacio(Izq(t)) }}\textsubscript{y que para todo \textit{a }subárbol de \textit{t }se verifica que \textit{Esvacio(Izq(a))}. Por tanto, el número de OE que se realizan en la ejecución de \textit{Altura(t) }para un árbol \textit{t }con \textit{n }elementos es: }\par}\par

\end{adjustwidth}

\begin{adjustwidth}{0.87in}{0.82in}
{\fontsize{10pt}{12.0pt}\selectfont \textit{T}(\textit{n})= (1+\textit{c}) + (1+\textit{c}+1+\textit{T}(0) + 1+\textit{c}+1+\textit{T}(\textit{n}–1)) +3 = 8 + 3\textit{c }+ \textit{T}(0) + \textit{T}(\textit{n}–1). \textit{\textsubscript{T}}\textsubscript{(0)= (1+\textit{c}) + 1 = 2+\textit{c}. }\par}\par

\end{adjustwidth}

\begin{adjustwidth}{0.77in}{0.57in}
{\fontsize{10pt}{12.0pt}\selectfont Con esto, \textit{T}(\textit{n})=10+4\textit{c}+\textit{T}(\textit{n}–1), ecuación en recurrencia no homogénea que \textsubscript{podemos resolver desarrollándola telescópicamente: }\par}\par

\end{adjustwidth}

\begin{adjustwidth}{1.17in}{1.2in}
{\fontsize{10pt}{12.0pt}\selectfont \textit{T}(\textit{n}) = 10 + 4\textit{c }+ \textit{T}(\textit{n}–1) = (10 + 4\textit{c}) + (10 + 4\textit{c}) + \textit{T}(\textit{n}–2) = ... = \textsubscript{(10 + 4\textit{c})\textit{n }+ (2 + \textit{c}) $ \in $  $ \Theta $ (\textit{n}) }\par}\par

\end{adjustwidth}

\begin{adjustwidth}{0.57in}{0.57in}
{\fontsize{10pt}{12.0pt}\selectfont $\bullet$  Si \textit{t }es equilibrado sus dos subárboles tienen del orden de \textit{n}/2 elementos y son \textsubscript{también equilibrados\textit{. }Por tanto, el número de OE que se realizan en la ejecución de \textit{Altura(t) }para un árbol \textit{t }con \textit{n }elementos es: }\par}\par

\end{adjustwidth}

\begin{adjustwidth}{1.17in}{0.77in}
{\fontsize{10pt}{12.0pt}\selectfont \textit{T}(\textit{n})= (1+\textit{c}) + (1+\textit{c}+1+\textit{T}(\textit{n}/2)) + 1+\textit{c}+1+\textit{T}(\textit{n}/2)) +3 = 8 + 3\textit{c }+ 2\textit{T}(\textit{n}/2). \textit{\textsubscript{T}}\textsubscript{(0)= 2+\textit{c}. }\par}\par

\end{adjustwidth}

\begin{adjustwidth}{0.58in}{0.57in}
{\fontsize{10pt}{12.0pt}\selectfont 48 {\fontsize{7pt}{8.4pt}\selectfont TÉCNICAS DE DISEÑO DE ALGORITMOS \par}\par}\par

\end{adjustwidth}

\begin{adjustwidth}{0.77in}{0.57in}
{\fontsize{10pt}{12.0pt}\selectfont Para resolver esta ecuación en recurrencia se hace el cambio \textit{t\textsubscript{k}}\textsubscript{=\textit{T}(2}{\fontsize{7pt}{8.4pt}\selectfont \textit{k}\textsubscript{), con lo que obtenemos }\par}\par}\par

\end{adjustwidth}

\begin{adjustwidth}{0.77in}{0.82in}
\begin{Center}
{\fontsize{10pt}{12.0pt}\selectfont \textit{t\textsubscript{k }}\textsubscript{– 2\textit{tk}–1 = 8 + 3\textit{c}, }ecuación no homogénea de ecuación característica (\textit{x}–2)(\textit{x}–1) = 0. Por tanto, \par}
\end{Center}\par

\end{adjustwidth}

\begin{adjustwidth}{0.77in}{2.74in}
{\fontsize{10pt}{12.0pt}\selectfont \textit{t\textsubscript{k }}\textsubscript{= \textit{c}12}{\fontsize{7pt}{8.4pt}\selectfont \textit{k }\textsubscript{+ \textit{c}2. }{\fontsize{10pt}{12.0pt}\selectfont Deshaciendo los cambios, \par}\par}\par}\par

\end{adjustwidth}

\begin{adjustwidth}{0.77in}{0.57in}
\begin{Center}
{\fontsize{10pt}{12.0pt}\selectfont \textit{T}(\textit{n}) = \textit{c}\textsubscript{1\textit{n }+ \textit{c}2. }Para calcular las constantes, nos apoyamos en la condición inicial \textit{\textsubscript{T}}\textsubscript{(0)=2+\textit{c}, junto con el valor de \textit{T}(1), que puede ser calculado basándonos en la expresión de la ecuación en recurrencia: \textit{T}(1) = 8 + 3\textit{c }+ 2(2 + \textit{c}). Finalmente obtenemos }\par}
\end{Center}\par

\end{adjustwidth}

\begin{adjustwidth}{2.06in}{2.06in}
{\fontsize{10pt}{12.0pt}\selectfont \textit{T}(\textit{n}) = (10 + 4\textit{c})\textit{n }+ (2 + \textit{c}) $ \in $  $ \Theta $ (\textit{n}). \par}\par

\end{adjustwidth}

\begin{adjustwidth}{0.57in}{0.57in}
\begin{justify}
{\fontsize{10pt}{12.0pt}\selectfont \textbf{Función \textit{Mezcla }}( ) Para resolver este problema vamos a suponer que el tiempo de ejecución del \textsubscript{procedimiento \textit{Ins}, que inserta un elemento en un árbol binario de búsqueda, es \textit{A}log\textit{n}+\textit{B}, siendo \textit{A }y \textit{B }dos constantes. Supongamos también que \textit{n }y \textit{m }son el número de elementos de \textit{t1 }y \textit{t2 }respectivamente. }Para estudiar el tiempo de ejecución \textit{T}(\textit{n},\textit{m}) consideraremos, al igual que \textsubscript{hicimos para la función anterior, dos casos extremos: que el árbol \textit{t2 }sea degenerado (es decir, una lista) o que sea equilibrado. }\par}
\end{justify}\par

\end{adjustwidth}

\begin{adjustwidth}{0.57in}{0.57in}
{\fontsize{10pt}{12.0pt}\selectfont $\bullet$  Si \textit{t2 }es degenerado, podemos suponer sin pérdida de generalidad que \textit{\textsubscript{Esvacio(Izq(t2)) }}\textsubscript{y que para todo \textit{a }subárbol de \textit{t2 }se verifica que \textit{Esvacio(Izq(a))}. Por tanto, vamos a ver el número de OE que se realizan en cada línea de la función en este caso\textit{: }}– En la línea (1) se invoca a \textit{Esvacio(t1)}, lo que supone 1+\textit{c }OE. – En la línea (2) se efectúa 1 OE. – Análogamente, las líneas (3) y (4) realizan (1+\textit{c}) y 1 respectivamente. – Para estudiar el número de OE que realiza la línea (6), vamos a dividirla en \textsubscript{cuatro partes: }a) \textit{a1:=Ins(t1,Raiz(t2))}, siendo \textit{a1 }una variable auxiliar para efectuar los \textsubscript{cálculos. Se efectúan 2+\textit{c}+\textit{A}log\textit{n}+\textit{B }operaciones elementales: la llamada a \textit{Raiz }(1), el tiempo que ésta tarda (\textit{c}), la llamada a \textit{Ins }(1 OE), y su tiempo de ejecución (\textit{A}log\textit{n}+\textit{B})\textit{. }}b) \textit{a2:=Mezcla(a1,Izq(t2))}, siendo \textit{a2 }una variable auxiliar para efectuar los \textsubscript{cálculos. Se efectúan aquí 2+\textit{c}+\textit{T}(\textit{n}+1,0) operaciones elementales: llamada a I\textit{zq }(1), el tiempo que ésta tarda (\textit{c}), la llamada a M\textit{ezcla }(1 OE), y su tiempo de ejecución, que será \textit{T}(\textit{n}+1,0), pues estamos suponiendo que \textit{Esvacio(Izq(a)) }para todo \textit{a }subárbol de \textit{t2}. }c) \textit{a3:=Mezcla(a2,Der(t2))}, siendo \textit{a3 }una variable auxiliar para efectuar los \textsubscript{cálculos. Se efectúan 2+\textit{c}+\textit{T}(\textit{n}+1,\textit{m}–1) operaciones elementales: la }\par}\par

\end{adjustwidth}

{\fontsize{7pt}{8.4pt}\selectfont LA COMPLEJIDAD DE LOS ALGORITMOS {\fontsize{10pt}{12.0pt}\selectfont 49 \par}\par}\par

{\fontsize{10pt}{12.0pt}\selectfont llamada a \textit{Der }(1), el tiempo que ésta tarda (\textit{c}), la llamada a \textit{Mezcla }\textsubscript{(1 suponiendo OE), y su que tiempo \textit{Esvacio(Izq(a)) }de ejecución, para que todo será \textit{a }subárbol \textit{T}(\textit{n}+1,\textit{m}–1), de \textit{t2 }pues o, lo estamos que es igual, que el número de elementos de \textit{Der(t) }es \textit{m}–1. }d) \textit{RETURN a3}, que realiza 1 OE. \par}\par

{\fontsize{10pt}{12.0pt}\selectfont Por tanto, la ejecución de \textit{Mezcla(t1,t2) }en este caso es : \par}\par

{\fontsize{10pt}{12.0pt}\selectfont \textit{T}(\textit{n},\textit{m}) = 9 + 5\textit{c }+ \textit{B }+ \textit{A}log\textit{n }+ \textit{T}(\textit{n}+1,0) + \textit{T}(\textit{n}+1,\textit{m}–1) \par}\par

{\fontsize{10pt}{12.0pt}\selectfont con las condiciones iniciales \textit{T}(0,\textit{m}) = 2 + \textit{c }y \textit{T}(\textit{n},0) = 3 + 2\textit{c}. Para resolver la \textsubscript{ecuación en recurrencia podemos expresarla como: }\par}\par

{\fontsize{10pt}{12.0pt}\selectfont \textit{T}(\textit{n},\textit{m}) = 12 + 7\textit{c }+ \textit{B }+ \textit{A}log\textit{n }+ \textit{T}(\textit{n}+1,\textit{m}–1) \par}\par

{\fontsize{10pt}{12.0pt}\selectfont haciendo uso de la segunda condición inicial. Desarrollando telescópicamente la \textsubscript{ecuación: }\par}\par

{\fontsize{10pt}{12.0pt}\selectfont \textit{T}(\textit{n},\textit{m}) = 12 + 7\textit{c }+ \textit{B }+ \textit{A}log\textit{n }+ \textit{T}(\textit{n}+1,\textit{m}–1) = \par}\par

{\fontsize{10pt}{12.0pt}\selectfont = (12+7\textit{c}+\textit{B}+\textit{A}log\textit{n}) + (12+7\textit{c}+\textit{B}+\textit{A}log(\textit{n}+1)) + \textit{T}(\textit{n}+2,\textit{m}–2) = ............... \par}\par

{\fontsize{10pt}{12.0pt}\selectfont = \textit{m }\par}\par

)712( \par

\textsubscript{+++ }\textit{Bc }⎛ \textsubscript{│⎝$ \sum $ }{\fontsize{7pt}{8.4pt}\selectfont \textit{m\textsubscript{i }}\textsubscript{= }{\fontsize{6pt}{7.2pt}\selectfont $-$ \textsubscript{0 }1\textit{mnTinA }log( \textsubscript{+ }() ⎞ │\textsubscript{⎠}++ )0, {\fontsize{10pt}{12.0pt}\selectfont = \par}\par}\par}\par

{\fontsize{10pt}{12.0pt}\selectfont = \textit{m }32)712( \par}\par

\textsubscript{+++++ }\textit{cBc A }⎛ \textsubscript{│⎝$ \sum $ }{\fontsize{7pt}{8.4pt}\selectfont \textit{m\textsubscript{i }}\textsubscript{= }\par}\par

{\fontsize{6pt}{7.2pt}\selectfont $-$ \textsubscript{0 }1log( \textit{in }\textsubscript{+ }) ⎞ │\textsubscript{⎠}{\fontsize{10pt}{12.0pt}\selectfont . Pero como log(\textit{n}+\textit{i}) $ \leq $  log(\textit{n}+\textit{m}) para todo 0 $ \leq $  \textit{i }$ \leq $  \textit{m}, \par}\par}\par

{\fontsize{10pt}{12.0pt}\selectfont \textit{T}(\textit{n},\textit{m}) $ \leq $  \textit{m}(12 + 7\textit{c }+ \textit{B}) + 2\textit{c }+ 3 + \textit{Am}log(\textit{n}+\textit{m}) $ \in $  O(\textit{m}log(\textit{n}+\textit{m})) \par}\par

{\fontsize{10pt}{12.0pt}\selectfont $\bullet$  El \textsubscript{análoga }segundo \textsubscript{que }\par}\par

{\fontsize{10pt}{12.0pt}\selectfont caso es que \textit{t2 }sea equilibrado, para el que se demuestra de forma \textit{T}(\textit{n},\textit{m}) $ \in $  O(\textit{m}log(\textit{n}+\textit{m})). \par}\par

{\fontsize{10pt}{12.0pt}\selectfont \textbf{Solución al Problema 1.14 }(☺) \par}\par

\begin{justify}
{\fontsize{10pt}{12.0pt}\selectfont Para comprobar que O(\textit{f})$ \subset $ O(\textit{g}), basta ver que \textit{lim }\textsubscript{$\infty$ $ \rightarrow $ \textit{n }}(\textit{f}(\textit{n})/\textit{g}(\textit{n})) = 0 en cada caso pues las funciones son continuas, lo que implica la existencia de los límites. De esta forma se obtiene la siguiente ordenación: \par}
\end{justify}\par

\begin{Center}
{\fontsize{10pt}{12.0pt}\selectfont O((1/3)\textit{\textsuperscript{n}}) $ \subset $  O(17) $ \subset $  O(loglog\textit{n}) $ \subset $  O(log\textit{n}) $ \subset $  O(log\textsuperscript{2}\textit{n}) $ \subset $  O( \textit{n }) $ \subset $  O( \textit{n }log\textsuperscript{2}\textit{n}) $ \subset $  O(\textit{n}/log\textit{n}) $ \subset $  O(\textit{n}) $ \subset $  O(\textit{n}\textsuperscript{2}) $ \subset $  O((3/2)\textit{\textsuperscript{n}}). \par}
\end{Center}\par

{\fontsize{10pt}{12.0pt}\selectfont \textbf{Solución al Problema 1.15 }( ) \par}\par

{\fontsize{10pt}{12.0pt}\selectfont 50 {\fontsize{7pt}{8.4pt}\selectfont TÉCNICAS DE DISEÑO DE ALGORITMOS \par}\par}\par

{\fontsize{10pt}{12.0pt}\selectfont Para resolver la ecuación \par}\par

\textit{nT }\par

)( \textsubscript{= }\textsuperscript{1 }\textit{\textsubscript{n }}⎛ \textsubscript{│⎝$ \sum $ }{\fontsize{7pt}{8.4pt}\selectfont \textit{n\textsubscript{i }}\textsubscript{=}{\fontsize{6pt}{7.2pt}\selectfont $-$ \textsubscript{0 }1\textit{iT })( ⎞ │\textsubscript{⎠}+ \textit{cn }{\fontsize{10pt}{12.0pt}\selectfont , \par}\par}\par}\par

{\fontsize{10pt}{12.0pt}\selectfont siendo \textit{T}(0) = 0, podemos reescribirla como: \par}\par

\textit{\textsuperscript{nnT }}\par

\textsuperscript{)( }={\fontsize{18pt}{21.6pt}\selectfont $ \sum $ {\fontsize{7pt}{8.4pt}\selectfont \textit{n}\textsuperscript{$-$ 1\textit{cniT })( }+ {\fontsize{6pt}{7.2pt}\selectfont 2 \textit{\textsubscript{i }}= 0 \textsuperscript{[1.7] }\par}\par}\par}\par

{\fontsize{10pt}{12.0pt}\selectfont Por otro lado, para \textit{n}–1 obtenemos: \par}\par

\textsuperscript{)1()1( \textit{nTn }}\par

\textsuperscript{$-$  =$-$  $ \sum $ }{\fontsize{7pt}{8.4pt}\selectfont \textit{n}\textsuperscript{$-$ 2\textit{nciT })1()( $-$ + }{\fontsize{6pt}{7.2pt}\selectfont 2 \textit{\textsubscript{i }}= 0 {\fontsize{10pt}{12.0pt}\selectfont [1.8] \par}\par}\par}\par

{\fontsize{10pt}{12.0pt}\selectfont Restando [1.7] y [1.8]: \par}\par

{\fontsize{10pt}{12.0pt}\selectfont \textit{nT}(\textit{n}) – \textit{nT}(\textit{n}–1) + \textit{T}(\textit{n}–1) = \textit{T}(\textit{n}–1) + \textit{c}(2\textit{n}–1) $ \Rightarrow $  \textit{nT}(\textit{n}) = \textit{nT}(\textit{n}–1) + \textit{c}(2\textit{n}–1) $ \Rightarrow $  \par}\par

{\fontsize{10pt}{12.0pt}\selectfont \textit{T}(\textit{n}) = \textit{T}(\textit{n}–1) + \textit{c}(2–1/\textit{n}). \par}\par

{\fontsize{10pt}{12.0pt}\selectfont Desarrollando telescópicamente la ecuación en recurrencia: \par}\par

{\fontsize{10pt}{12.0pt}\selectfont \textit{T}(\textit{n}) = \textit{T}(\textit{n}–1) + \textit{c}(2 – 1/\textit{n}) = \par}\par

{\fontsize{10pt}{12.0pt}\selectfont = \textit{T}(\textit{n}–2) + \textit{c}(2 – 1/(\textit{n}–1)) + \textit{c}(2 – 1/\textit{n}) = = \textit{T}(\textit{n}–3) + \textit{c}(2 – 1/(\textit{n}–2)) + \textit{c}(2 – 1/(\textit{n}–1)) + \textit{c}(2 – 1/\textit{n})= .......... \par}\par

{\fontsize{10pt}{12.0pt}\selectfont = \textit{T }\par}\par

\textsubscript{)0( + }\par

\textit{c }{\fontsize{18pt}{21.6pt}\selectfont $ \sum $ {\fontsize{7pt}{8.4pt}\selectfont \textit{n\textsubscript{i }}\textsubscript{= }{\fontsize{6pt}{7.2pt}\selectfont 1 \par}\par}\par}\par

⎛ \textsubscript{│⎝2 $-$  }1 \textit{\textsubscript{i }}⎞ │\textsubscript{⎠= }\par

{\fontsize{10pt}{12.0pt}\selectfont = \textit{c }\par}\par

{\fontsize{18pt}{21.6pt}\selectfont $ \sum $ {\fontsize{7pt}{8.4pt}\selectfont \textit{n\textsubscript{i }}\textsubscript{= }{\fontsize{6pt}{7.2pt}\selectfont 1 \par}\par}\par}\par

⎛ \textsubscript{│⎝2 }\par

\begin{Center}
\textsubscript{$-$  }1 \textit{\textsubscript{i }}⎞ │\textsubscript{⎠}{\fontsize{10pt}{12.0pt}\selectfont ya que teníamos que \textit{T}(0) = 0. Veamos cual es el orden de \textit{T}(\textit{n}): \par}
\end{Center}\par

{\fontsize{10pt}{12.0pt}\selectfont a) Como (2–1/\textit{i}) $ \leq $  2 para todo \textit{i }> 0, \textit{T}(\textit{n}) $ \leq $  \textit{c }\par}\par

{\fontsize{18pt}{21.6pt}\selectfont $ \sum $ {\fontsize{7pt}{8.4pt}\selectfont \textit{n}\textsuperscript{2 = 2\textit{cn }$ \Rightarrow $  \textit{T}(\textit{n})$ \in $ O(\textit{n}). }\textit{\textsubscript{i}}\textsubscript{=}{\fontsize{6pt}{7.2pt}\selectfont 1 {\fontsize{10pt}{12.0pt}\selectfont b) Como (2–1/\textit{i}) $ \geq $  1 para todo \textit{i }> 0, \textit{T}(\textit{n}) $ \geq $  \textit{c }\par}\par}\par}\par}\par

{\fontsize{18pt}{21.6pt}\selectfont $ \sum $ {\fontsize{7pt}{8.4pt}\selectfont \textit{n}\textsuperscript{1 = \textit{cn }$ \Rightarrow $  \textit{T}(\textit{n})$ \in $ $ \Omega $ (\textit{n}). }\textit{\textsubscript{i}}\textsubscript{=}{\fontsize{6pt}{7.2pt}\selectfont 1{\fontsize{10pt}{12.0pt}\selectfont Por tanto, \textit{T}(\textit{n})$ \in $ $ \Theta $ (\textit{n}). \par}\par}\par}\par}\par

{\fontsize{10pt}{12.0pt}\selectfont \textbf{Solución al Problema 1.16 }\par}\par

\begin{adjustwidth}{0.58in}{0.57in}
{\fontsize{7pt}{8.4pt}\selectfont LA COMPLEJIDAD DE LOS ALGORITMOS {\fontsize{10pt}{12.0pt}\selectfont 51 \par}\par}\par

\end{adjustwidth}

\begin{adjustwidth}{0.58in}{0.57in}
{\fontsize{10pt}{12.0pt}\selectfont \textbf{Función \textit{BuscBin }}( ) a) Para determinar su tiempo de ejecución, calcularemos primero el número de \textsubscript{operaciones elementales (OE) que se realizan: }– En la línea (1) se ejecutan la comparación del \textit{IF }(1 OE), y un acceso a un \textsubscript{vector (1 OE), una comparación (1 OE) y un \textit{RETURN }(1 OE) si la condición es verdadera. }– En la línea (3) se realizan 3 OE (suma, división y asignación). – En la línea (4) hay un acceso a un vector (1 OE) y una comparación (1 OE), \textsubscript{y además 1 OE en caso de que la condición del \textit{IF }sea verdadera. }– En la línea (5) hay un acceso a un vector (1 OE) y una comparación (1 OE). – Las líneas (6) y (8) efectúan 3+\textit{T}(\textit{n}/2) cada una: una operación aritmética \textsubscript{(incremento o decremento de 1), una llamada a la función \textit{BuscBin }(lo que supone 1 OE), más lo que tarde en ejecutarse la función con la mitad de los elementos y un \textit{RETURN }(1 OE). }Por tanto obtenemos la ecuación en recurrencia \textit{T}(\textit{n}) = 11 + \textit{T}(\textit{n}/2), con la \textsubscript{condición inicial \textit{T}(1) = 4. Para resolverla, haciendo el cambio \textit{tk }= \textit{T}(2}{\fontsize{7pt}{8.4pt}\selectfont \textit{k}\textsubscript{) obtenemos }\par}\par}\par

\end{adjustwidth}

\begin{adjustwidth}{0.77in}{0.8in}
\begin{Center}
{\fontsize{10pt}{12.0pt}\selectfont \textit{t\textsubscript{k }}\textsubscript{– \textit{tk}–1 = 11, }ecuación no homogénea cuya ecuación característica es (\textit{x}–1)\textsuperscript{2 }= 0. Por tanto, \par}
\end{Center}\par

\end{adjustwidth}

\begin{adjustwidth}{0.77in}{2.76in}
{\fontsize{10pt}{12.0pt}\selectfont \textit{t\textsubscript{k }}\textsubscript{= \textit{c}1\textit{k }+ \textit{c}2 }y, deshaciendo los cambios, \par}\par

\end{adjustwidth}

\begin{adjustwidth}{0.77in}{0.57in}
\begin{Center}
{\fontsize{10pt}{12.0pt}\selectfont \textit{T}(\textit{n}) = \textit{c}\textsubscript{1log\textit{n }+ \textit{c}2. }Para calcular las constantes, nos basaremos en la condición inicial \textit{T}(1) = 4, \textsubscript{junto con el valor de \textit{T}(2), que podemos calcular apoyándonos en la expresión de la ecuación en recurrencia: \textit{T}(2) = 11 + 4 = 15. Finalmente obtenemos }\par}
\end{Center}\par

\end{adjustwidth}

\begin{adjustwidth}{2.25in}{2.25in}
{\fontsize{10pt}{12.0pt}\selectfont \textit{T}(\textit{n}) = 11log\textit{n }+ 4 $ \in $  $ \Theta $ (log\textit{n}) \par}\par

\end{adjustwidth}

\begin{adjustwidth}{0.58in}{0.57in}
{\fontsize{10pt}{12.0pt}\selectfont b) La recursión de este programa, por tratarse de un caso de recursión de cola, \textsubscript{puede ser eliminada mediante un bucle que simule las llamadas recursivas a la función. La condición de terminación del bucle puede ser tomada del caso base de la función recursiva y el cuerpo de dicho bucle consiste en una preparación de los argumentos de la función recursiva y el cálculo que ésta realiza: }\par}\par

\end{adjustwidth}

\begin{adjustwidth}{0.97in}{0.66in}
{\fontsize{10pt}{12.0pt}\selectfont PROCEDURE BuscBIt(a:vector;prim,ult:CARDINAL;x:INTEGER):BOOLEAN; \par}\par

\end{adjustwidth}

\begin{adjustwidth}{0.97in}{3.72in}
{\fontsize{10pt}{12.0pt}\selectfont VAR mitad:CARDINAL; BEGIN \par}\par

\end{adjustwidth}

\begin{adjustwidth}{1.17in}{1.84in}
{\fontsize{10pt}{12.0pt}\selectfont WHILE (prim<ult) DO ($\ast$  1 $\ast$ ) \par}\par

\end{adjustwidth}

\begin{adjustwidth}{1.36in}{1.74in}
\begin{FlushRight}
{\fontsize{10pt}{12.0pt}\selectfont mitad:=(prim+ult)DIV 2; ($\ast$  2 $\ast$ ) IF x=a[mitad] THEN RETURN TRUE ($\ast$  3 $\ast$ ) ELSIF (x<a[mitad]) THEN ($\ast$  4 $\ast$ ) ult:=mitad-1 ($\ast$  5 $\ast$ ) ELSE ($\ast$  6 $\ast$ ) \par}
\end{FlushRight}\par

\end{adjustwidth}

{\fontsize{10pt}{12.0pt}\selectfont 52 {\fontsize{7pt}{8.4pt}\selectfont TÉCNICAS DE DISEÑO DE ALGORITMOS \par}\par}\par

\begin{FlushRight}
{\fontsize{10pt}{12.0pt}\selectfont prim:=mitad+1 ($\ast$  7 $\ast$ ) END ($\ast$  8 $\ast$ ) END; ($\ast$  9 $\ast$ ) RETURN x=a[ult] ($\ast$  10 $\ast$ ) END BuscBIt; \par}
\end{FlushRight}\par

\begin{justify}
{\fontsize{10pt}{12.0pt}\selectfont c) Para el cálculo del tiempo de ejecución y la complejidad de la función no \textsubscript{recursiva podemos seguir un proceso análogo al que seguimos para la función \textit{Algoritmo2 }(en el problema 1.5). Para determinar el tiempo de ejecución, calcularemos realizan en cada primero una de las el número líneas: de operaciones elementales (OE) que se }– En \textsubscript{comparación). }la línea (1) se efectúa la condición del bucle, que supone 1 OE (la – Las \textsubscript{2, 0, }líneas \textsubscript{2, 0 y }(2) \textsubscript{0 OE }a (9) \textsubscript{respectivamente. }componen el cuerpo del bucle, y contabilizan 3, 2+1, 2, – Por \textsubscript{del bucle }último, \textsubscript{deja }la \textsubscript{de }línea \textsubscript{verificarse. }(10) supone 3 OE. A ella se llega cuando la condición \textsubscript{ejecutarse }El bucle \textsubscript{la }se \textsubscript{línea }repite \textsubscript{(10). }hasta \textsubscript{Cada }que \textsubscript{iteración }su condición \textsubscript{del bucle }sea \textsubscript{está }falsa, \textsubscript{compuesta }acabando \textsubscript{por }la función \textsubscript{las líneas }al \textsubscript{(1) a (9), junto con una ejecución adicional de la línea (1) que es la que ocasiona la salida del bucle. En cada iteración se reduce a la mitad los elementos a considerar, por lo que el bucle se repite log\textit{n }veces. Por tanto, en el peor caso, }\par}
\end{justify}\par

{\fontsize{10pt}{12.0pt}\selectfont \textit{T}(\textit{n}) = \textsuperscript{⎛ │}{\fontsize{11pt}{13.2pt}\selectfont │⎝{\fontsize{6pt}{7.2pt}\selectfont log{\fontsize{11pt}{13.2pt}\selectfont log1031)22231( \par}\par}\par}\par}\par

\textsubscript{1 }{\fontsize{11pt}{13.2pt}\selectfont 4 ⎛ │\textsubscript{│⎝$ \sum $ = ++++ }⎞ │\textsubscript{│⎠}+ ⎞ \par}\par

{\fontsize{11pt}{13.2pt}\selectfont │\textsubscript{│⎠}{\fontsize{7pt}{8.4pt}\selectfont \textit{n }\par}\par}\par

{\fontsize{11pt}{13.2pt}\selectfont =+ \textit{n }+ {\fontsize{7pt}{8.4pt}\selectfont \textit{i }\par}\par}\par

{\fontsize{10pt}{12.0pt}\selectfont $ \in $ $ \Theta $ (log\textit{n}). \par}\par

{\fontsize{10pt}{12.0pt}\selectfont Como puede verse, el tiempo de ejecución de ambas funciones es \textsubscript{prácticamente igual, lo que a priori implica que cualquiera de las dos pueden usarse complejidad indistintamente. espacial que Sin siempre embargo, suponen hay los que procedimientos tener en cuenta recursivos la mayor por la utilización de la pila, lo que hace que ante una igualdad de tiempos de ejecución, los procedimientos iterativos sean preferibles frente a los recursivos. Pero algoritmo. no sólo La la claridad complejidad y sencillez ha de del ser código tenida es en un cuenta factor también para la a elección considerar, del pues ello va a implicar una mejor legibilidad y una depuración del programa y mantenimiento más fácil, aspectos todos ellos muy importantes. }\par}\par

{\fontsize{10pt}{12.0pt}\selectfont \textbf{Función \textit{Sumadigitos }}( ) \par}\par

\begin{justify}
{\fontsize{10pt}{12.0pt}\selectfont a) Para calcular el tiempo de ejecución, calcularemos primero el número de \textsubscript{operaciones elementales (OE) que se realizan: }– En la línea (1) se ejecutan una comparación (1 OE) y un \textit{RETURN }(1 OE) si \textsubscript{la condición es verdadera. }– En la línea (2) se efectúa una división (1 OE), una llamada a la función \textit{\textsubscript{Sumadigitos }}\textsubscript{(1 OE), más lo que tarda ésta con un décimo del tamaño de su entrada, una suma (1 OE), un resto (1 OE), y un \textit{RETURN }(1 OE). }\par}
\end{justify}\par

\begin{adjustwidth}{0.58in}{0.57in}
{\fontsize{7pt}{8.4pt}\selectfont LA COMPLEJIDAD DE LOS ALGORITMOS {\fontsize{10pt}{12.0pt}\selectfont 53 \par}\par}\par

\end{adjustwidth}

\begin{adjustwidth}{0.77in}{0.57in}
{\fontsize{10pt}{12.0pt}\selectfont Llamando \textit{n }al parámetro \textit{num }de la función, obtenemos la ecuación en \textsubscript{recurrencia \textit{T}(\textit{n}) = 6 + \textit{T}(\textit{n}/10), con la condición inicial \textit{T}(1) = 2. }Para resolverla hacemos los cambios \textit{n }= 10\textit{\textsuperscript{k }}(o, lo que es igual, \textit{k }= log\textsubscript{10\textit{n}) y \textit{tk }= \textit{T}(10}{\fontsize{7pt}{8.4pt}\selectfont \textit{k}\textsubscript{) y obtenemos }\par}\par}\par

\end{adjustwidth}

\begin{adjustwidth}{0.77in}{0.8in}
\begin{Center}
{\fontsize{10pt}{12.0pt}\selectfont \textit{t\textsubscript{k }}\textsubscript{– \textit{tk}–1 = 6, }ecuación no homogénea cuya ecuación característica es (\textit{x}–1)\textsuperscript{2 }= 0. Por tanto, \par}
\end{Center}\par

\end{adjustwidth}

\begin{adjustwidth}{0.77in}{2.74in}
{\fontsize{10pt}{12.0pt}\selectfont \textit{t\textsubscript{k }}\textsubscript{= \textit{c}1 + \textit{c}2\textit{k}. }Deshaciendo los cambios, \par}\par

\end{adjustwidth}

\begin{adjustwidth}{0.77in}{0.57in}
\begin{Center}
{\fontsize{10pt}{12.0pt}\selectfont \textit{T}(\textit{n}) = \textit{c}\textsubscript{1 + \textit{c}2log10\textit{n}. }Para calcular las constantes, nos apoyamos en la condición inicial \textit{T}(1) = 2, \textsubscript{junto con el valor de \textit{T}(10), que puede ser calculado apoyándonos en la expresión de la ecuación en recurrencia: \textit{T}(10) = 6 + 2 = 8. Finalmente obtenemos }\par}
\end{Center}\par

\end{adjustwidth}

\begin{adjustwidth}{0.77in}{0.57in}
\begin{Center}
{\fontsize{10pt}{12.0pt}\selectfont \textit{T}(\textit{n}) = 6 log\textsubscript{10\textit{n }+ 2 $ \in $  $ \Theta $ (log\textit{n}) }Como vemos, en esta caso la complejidad de la función depende del \textsubscript{logaritmo en base 10 de su parámetro \textit{num }(esto es, de su número de dígitos). }\par}
\end{Center}\par

\end{adjustwidth}

\begin{adjustwidth}{0.58in}{0.57in}
{\fontsize{10pt}{12.0pt}\selectfont b) La recursión de este algoritmo puede ser eliminada mediante un bucle que \textsubscript{simule las llamadas recursivas a la función, cuya condición de terminación puede ser tomada del caso base de la función recursiva, y cuyo cuerpo consiste en los cálculos que ésta realiza, junto con una preparación de los argumentos de la siguiente llamada. En concreto, el algoritmo que implementa el algoritmo no recursivo es el siguiente: }\par}\par

\end{adjustwidth}

\begin{adjustwidth}{1.46in}{1.32in}
{\fontsize{10pt}{12.0pt}\selectfont PROCEDURE Sumadigitos\_it(num:CARDINAL):CARDINAL; \par}\par

\end{adjustwidth}

\begin{adjustwidth}{1.46in}{1.44in}
{\fontsize{10pt}{12.0pt}\selectfont VAR s:CARDINAL; BEGIN \textsubscript{s:=num MOD 10; ($\ast$  1 $\ast$ ) }WHILE num>=10 DO ($\ast$  2 $\ast$ ) num:=num DIV 10; ($\ast$  3 $\ast$ ) s:=s+(num MOD 10) ($\ast$  4 $\ast$ ) END; ($\ast$  5 $\ast$ ) RETURN s ($\ast$  6 $\ast$ ) END Sumadigitos\_it; \par}\par

\end{adjustwidth}

\begin{adjustwidth}{0.58in}{0.57in}
{\fontsize{10pt}{12.0pt}\selectfont c) Para determinar el tiempo de ejecución, calcularemos primero el número de \textsubscript{operaciones elementales (OE) que se realizan: }\par}\par

\end{adjustwidth}


\vspace{\baselineskip}

\printbibliography
\end{document}